%%
%% Automatically generated file from DocOnce source
%% (https://github.com/hplgit/doconce/)
%%
% #ifdef PTEX2TEX_EXPLANATION
%%
%% The file follows the ptex2tex extended LaTeX format, see
%% ptex2tex: http://code.google.com/p/ptex2tex/
%%
%% Run
%%      ptex2tex myfile
%% or
%%      doconce ptex2tex myfile
%%
%% to turn myfile.p.tex into an ordinary LaTeX file myfile.tex.
%% (The ptex2tex program: http://code.google.com/p/ptex2tex)
%% Many preprocess options can be added to ptex2tex or doconce ptex2tex
%%
%%      ptex2tex -DMINTED myfile
%%      doconce ptex2tex myfile envir=minted
%%
%% ptex2tex will typeset code environments according to a global or local
%% .ptex2tex.cfg configure file. doconce ptex2tex will typeset code
%% according to options on the command line (just type doconce ptex2tex to
%% see examples). If doconce ptex2tex has envir=minted, it enables the
%% minted style without needing -DMINTED.
% #endif

% #define PREAMBLE

% #ifdef PREAMBLE
%-------------------- begin preamble ----------------------

\documentclass[%
twoside,                 % oneside: electronic viewing, twoside: printing
final,                   % or draft (marks overfull hboxes, figures with paths)
10pt]{article}

\listfiles               % print all files needed to compile this document

\usepackage{relsize,epsfig,makeidx,color,setspace,amsmath,amsfonts}
\usepackage[table]{xcolor}
\usepackage{bm,microtype}

\usepackage{ptex2tex}

\usepackage[T1]{fontenc}
%\usepackage[latin1]{inputenc}
\usepackage[utf8]{inputenc}

\usepackage{lmodern}         % Latin Modern fonts derived from Computer Modern

% Hyperlinks in PDF:
\definecolor{linkcolor}{rgb}{0,0,0.4}
\usepackage[%
    colorlinks=true,
    linkcolor=linkcolor,
    urlcolor=linkcolor,
    citecolor=black,
    filecolor=black,
    %filecolor=blue,
    pdfmenubar=true,
    pdftoolbar=true,
    bookmarksdepth=3   % Uncomment (and tweak) for PDF bookmarks with more levels than the TOC
            ]{hyperref}
%\hyperbaseurl{}   % hyperlinks are relative to this root

\setcounter{tocdepth}{2}  % number chapter, section, subsection

% Tricks for having figures close to where they are defined:
% 1. define less restrictive rules for where to put figures
\setcounter{topnumber}{2}
\setcounter{bottomnumber}{2}
\setcounter{totalnumber}{4}
\renewcommand{\topfraction}{0.85}
\renewcommand{\bottomfraction}{0.85}
\renewcommand{\textfraction}{0.15}
\renewcommand{\floatpagefraction}{0.7}
% 2. ensure all figures are flushed before next section
\usepackage[section]{placeins}
% 3. enable begin{figure}[H] (often leads to ugly pagebreaks)
%\usepackage{float}\restylefloat{figure}

\usepackage[framemethod=TikZ]{mdframed}

% --- begin definitions of admonition environments ---

% Admonition style "mdfbox" is an oval colored box based on mdframed
% "notice" admon
\colorlet{mdfbox_notice_background}{gray!5}
\newmdenv[
  skipabove=15pt,
  skipbelow=15pt,
  outerlinewidth=0,
  backgroundcolor=mdfbox_notice_background,
  linecolor=black,
  linewidth=2pt,       % frame thickness
  frametitlebackgroundcolor=mdfbox_notice_background,
  frametitlerule=true,
  frametitlefont=\normalfont\bfseries,
  shadow=false,        % frame shadow?
  shadowsize=11pt,
  leftmargin=0,
  rightmargin=0,
  roundcorner=5,
  needspace=0pt,
]{notice_mdfboxmdframed}

\newenvironment{notice_mdfboxadmon}[1][]{
\begin{notice_mdfboxmdframed}[frametitle=#1]
}
{
\end{notice_mdfboxmdframed}
}

% Admonition style "mdfbox" is an oval colored box based on mdframed
% "summary" admon
\colorlet{mdfbox_summary_background}{gray!5}
\newmdenv[
  skipabove=15pt,
  skipbelow=15pt,
  outerlinewidth=0,
  backgroundcolor=mdfbox_summary_background,
  linecolor=black,
  linewidth=2pt,       % frame thickness
  frametitlebackgroundcolor=mdfbox_summary_background,
  frametitlerule=true,
  frametitlefont=\normalfont\bfseries,
  shadow=false,        % frame shadow?
  shadowsize=11pt,
  leftmargin=0,
  rightmargin=0,
  roundcorner=5,
  needspace=0pt,
]{summary_mdfboxmdframed}

\newenvironment{summary_mdfboxadmon}[1][]{
\begin{summary_mdfboxmdframed}[frametitle=#1]
}
{
\end{summary_mdfboxmdframed}
}

% Admonition style "mdfbox" is an oval colored box based on mdframed
% "warning" admon
\colorlet{mdfbox_warning_background}{gray!5}
\newmdenv[
  skipabove=15pt,
  skipbelow=15pt,
  outerlinewidth=0,
  backgroundcolor=mdfbox_warning_background,
  linecolor=black,
  linewidth=2pt,       % frame thickness
  frametitlebackgroundcolor=mdfbox_warning_background,
  frametitlerule=true,
  frametitlefont=\normalfont\bfseries,
  shadow=false,        % frame shadow?
  shadowsize=11pt,
  leftmargin=0,
  rightmargin=0,
  roundcorner=5,
  needspace=0pt,
]{warning_mdfboxmdframed}

\newenvironment{warning_mdfboxadmon}[1][]{
\begin{warning_mdfboxmdframed}[frametitle=#1]
}
{
\end{warning_mdfboxmdframed}
}

% Admonition style "mdfbox" is an oval colored box based on mdframed
% "question" admon
\colorlet{mdfbox_question_background}{gray!5}
\newmdenv[
  skipabove=15pt,
  skipbelow=15pt,
  outerlinewidth=0,
  backgroundcolor=mdfbox_question_background,
  linecolor=black,
  linewidth=2pt,       % frame thickness
  frametitlebackgroundcolor=mdfbox_question_background,
  frametitlerule=true,
  frametitlefont=\normalfont\bfseries,
  shadow=false,        % frame shadow?
  shadowsize=11pt,
  leftmargin=0,
  rightmargin=0,
  roundcorner=5,
  needspace=0pt,
]{question_mdfboxmdframed}

\newenvironment{question_mdfboxadmon}[1][]{
\begin{question_mdfboxmdframed}[frametitle=#1]
}
{
\end{question_mdfboxmdframed}
}

% Admonition style "mdfbox" is an oval colored box based on mdframed
% "block" admon
\colorlet{mdfbox_block_background}{gray!5}
\newmdenv[
  skipabove=15pt,
  skipbelow=15pt,
  outerlinewidth=0,
  backgroundcolor=mdfbox_block_background,
  linecolor=black,
  linewidth=2pt,       % frame thickness
  frametitlebackgroundcolor=mdfbox_block_background,
  frametitlerule=true,
  frametitlefont=\normalfont\bfseries,
  shadow=false,        % frame shadow?
  shadowsize=11pt,
  leftmargin=0,
  rightmargin=0,
  roundcorner=5,
  needspace=0pt,
]{block_mdfboxmdframed}

\newenvironment{block_mdfboxadmon}[1][]{
\begin{block_mdfboxmdframed}[frametitle=#1]
}
{
\end{block_mdfboxmdframed}
}

% --- end of definitions of admonition environments ---

% prevent orhpans and widows
\clubpenalty = 10000
\widowpenalty = 10000

% --- end of standard preamble for documents ---


% insert custom LaTeX commands...

\raggedbottom
\makeindex

%-------------------- end preamble ----------------------

\begin{document}

% #endif


% ------------------- main content ----------------------



% ----------------- title -------------------------

\title{Slides from FYS3150/4150 Lectures, introduction to the course}

% ----------------- author(s) -------------------------

\author{Morten Hjorth-Jensen\inst{1,2}}
\institute{Department of Physics, University of Oslo\inst{1}
\and
Department of Physics and Astronomy and National Superconducting Cyclotron Laboratory, Michigan State University\inst{2}}
% ----------------- end author(s) -------------------------

\date{Fall 2015
% <optional titlepage figure>
}

% !split
\subsection{Overview of week 34}

\begin{block}{}
\begin{itemize}
  \item Thursday: First lecture: Presentation of the course, aims and content

  \item Thursday: Second Lecture: Introduction to C++ programming and numerical precision.

  \item Friday: Numerical precision and C++ programming, continued and exercises for first week

  \item Numerical differentiation and loss of numerical precision (chapter 3 lecture notes)

  \item Computer lab: Thursday and Friday. First time: Thursday and Friday this week, Presentation of hardware and software at room FV329 first hour of every labgroup and solution of first simple exercises. The first two weeks we focus on simple programming exercises and to set up github and QTcreator.
\end{itemize}

\noindent
\end{block}

% !split
\subsection{Lectures and ComputerLab}

\begin{block}{}
\begin{itemize}
  \item Lectures: Thursday (8.15am-10am) and Friday (8.15am-10am).

  \item Weekly reading assignments needed to solve projects.

  \item First hour of each lab session may be used to discuss technicalities, address questions etc linked with projects.

  \item Detailed lecture notes, exercises, all programs presented, projects etc can be found at the homepage of the course.

  \item Computerlab: Thursday (10am-6pm) and Friday (10am-6pm) room FV329.

  \item Weekly plans and all other information are on the official webpage.

  \item Final written exam December 14, 2.30pm (four hours).
\end{itemize}

\noindent
\end{block}

% !split
\subsection{Course Format}

\begin{block}{}
\begin{itemize}
  \item Several computer exercises, 5 compulsory projects. Electronic reports only.

  \item Evaluation and grading: The last project (50% of final grade) and a  final written exam (50% of final grade). Final written exam December 14.

  \item The computer lab (room FV329)consists of 16 Linux PCs, but many prefer own laptops. C/C++ is the default programming language, but Fortran2008 and Python are also used. All source codes discussed during the lectures can be found at the webpage of the course. We recommend either C/C++, Fortran2008 or Python as languages.
\end{itemize}

\noindent
\end{block}

% !split
\subsection{ComputerLab}

\begin{block}{}


{\footnotesize
\begin{tabular}{ll}
\hline
\multicolumn{1}{c}{ day } & \multicolumn{1}{c}{ teacher } \\
\hline
Thursday 10am-2pm & Anders, Morten L., Håvard, MHJ \\
Thursday 2pm-6pm  & Anders, Morten L., Håvard, MHJ \\
Friday 10am-2pm   & Anders, Morten L., Håvard, MHJ \\
Friday 2pm-6pm    & Anders, Morten L., Håvard, MHJ \\
\hline
\end{tabular}
}

\noindent
\end{block}

% !split
\subsection{Topics covered in this course}

\begin{block}{}
\begin{itemize}
  \item Numerical precision and intro to C++ programming

  \item Numerical derivation and integration

  \item Random numbers and Monte Carlo integration

  \item Monte Carlo methods in statistical physics

  \item Quantum Monte Carlo methods

  \item Linear algebra and eigenvalue problems

  \item Non-linear equations and roots of polynomials

  \item Ordinary differential equations

  \item Partial differential equations

  \item Parallelization of codes

  \item Programming av GPUs (optional)
\end{itemize}

\noindent
\end{block}


% !split
\subsection{Syllabus FYS3150}

\begin{block}{Linear algebra and eigenvalue problems, chapters 6 and 7 }
\begin{itemize}
  \item Know Gaussian elimination and LU decomposition

  \item How to solve linear equations

  \item How to obtain the inverse and the determinant of a real symmetric matrix

  \item Cholesky and tridiagonal matrix decomposition
\end{itemize}

\noindent
\end{block}


% !split
\subsection{Syllabus FYS3150}

\begin{block}{Linear algebra and eigenvalue problems, chapters 6 and 7 }
\begin{itemize}
  \item Householder's tridiagonalization technique and finding eigenvalues based on this

  \item Jacobi's method for finding eigenvalues

  \item Singular value decomposition

  \item Qubic Spline interpolation
\end{itemize}

\noindent
\end{block}


% !split
\subsection{Syllabus FYS3150}

\begin{block}{Numerical integration, standard methods and Monte Carlo methods (chapters 4 and 11) }
\begin{itemize}
  \item Trapezoidal, rectangle and Simpson's rules

  \item Gaussian quadrature, emphasis on Legendre polynomials, but you need to know about other polynomials as well.

  \item Brute force Monte Carlo integration

  \item Random numbers (simplest algo, ran0) and probability distribution functions, expectation values

  \item Improved Monte Carlo integration and importance sampling.
\end{itemize}

\noindent
\end{block}


% !split
\subsection{Syllabus FYS3150}

\begin{block}{Monte Carlo methods in physics (chapters 12, 13, and 14) }
\begin{itemize}
  \item Random walks and Markov chains and relation with diffusion equation

  \item Metropolis algorithm, detailed balance and ergodicity

  \item Simple spin systems and phase transitions

  \item Variational Monte Carlo

  \item How to construct trial wave functions for quantum systems
\end{itemize}

\noindent
\end{block}


% !split
\subsection{Syllabus FYS3150}

\begin{block}{Ordinary differential equations (chapters 8 and 9) }
\begin{itemize}
  \item Euler's method and improved Euler's method, truncation errors

  \item Runge Kutta methods, 2nd and 4th order, truncation errors

  \item How to implement a second-order differential equation, both linear and non-linear. How to make your equations dimensionless.

  \item Boundary value problems, shooting and matching method (chap 9).
\end{itemize}

\noindent
\end{block}


% !split
\subsection{Syllabus FYS3150}

\begin{block}{Partial differential equations, chapter 10 }
\begin{itemize}
  \item Set up diffusion, Poisson and wave equations up to 2 spatial dimensions and time

  \item Set up the mathematical model and algorithms for these equations, with boundary and initial conditions. Their stability conditions.

  \item Explicit, implicit and Crank-Nicolson schemes, and how to solve them. Remember that they result in triangular matrices.

  \item How to compute the Laplacian in Poisson's equation.

  \item How to solve the wave equation in one and two dimensions.
\end{itemize}

\noindent
\end{block}


% !split
\subsection{Overarching aims of this course}

\begin{itemize}
  \item Develop a critical approach to all steps in a project, which methods are most relevant, which natural laws and physical processes are important. Sort out initial conditions and boundary conditions etc.

  \item This means to teach you structured scientific computing, learn to structure a project.

  \item A critical understanding of central mathematical algorithms and methods from numerical analysis. In particular their limits and stability criteria.

  \item Always try to find good checks of your codes (like solutions on closed form)

  \item To enable you to develop a critical view on the mathematical model and the physics.
\end{itemize}

\noindent
% !split
\subsection{And, there is nothing like a code which gives correct results!!}



% inline figure
\centerline{\includegraphics[width=0.6\linewidth]{fig-intro/Nebbdyr2.pdf}}



\begin{itemize}
 \item J. J. Barton and L. R. Nackman,*Scientific and Engineering C++*, Addison Wesley, 3rd edition 2000.

 \item B. Stoustrup, \emph{The C++ programming language}, Pearson, 1997.

 \item H. P. Langtangen INF-VERK3830 \href{{http://heim.ifi.uio.no/~hpl/INF-VERK4830/}}{\nolinkurl{http://heim.ifi.uio.no/~hpl/INF-VERK4830/}}

 \item D. Yang, \emph{C++ and Object-oriented Numeric Computing for Scientists and Engineers}, Springer 2000.

 \item More books reviewed at \href{{http:://www.accu.org/}}{\nolinkurl{http:://www.accu.org/}} and \href{{http://www.comeaucomputing.com/booklist/}}{\nolinkurl{http://www.comeaucomputing.com/booklist/}}
\end{itemize}

\noindent
% !split
\subsection{Other courses in Computational Science at UiO}

\begin{block}{Bachelor/Master/PhD Courses }
\begin{itemize}
  \item INF-MAT4350 Numerical linear algebra

  \item MAT-INF3300/3310, PDEs and Sobolev spaces I and II

  \item INF-MAT3360 PDEs

  \item INF5620 Numerical methods for PDEs, finite element method

  \item FYS4411 Computational physics II (Parallelization (MPI), object orientation, quantum mechanical systems with many interacting particles), spring semester

  \item FYS4460 Computational physics III (Parallelization (MPI), object orientation, classical statistical physics, simulation of phase transitions, spring semester

  \item INF3331 Problem solving with high-level languages (Python), fall semester

  \item INF3380 Parallel computing for problems in the Natural Sciences (mostly PDEs), spring semester
\end{itemize}

\noindent
\end{block}


% !split
\subsection{Extremely useful tools, strongly recommended}

\begin{block}{and discussed at the lab sessions the first week }
\begin{itemize}
  \item GIT for version control (see webpage)

  \item ipython notebook

  \item QTcreator for editing and mastering computational projects (for C++ codes, see webpage of course)

  \item Armadillo as a useful numerical library for C++, highly recommended

  \item Unit tests, see also webpage

  \item Devilry for handing in projects
\end{itemize}

\noindent
\end{block}

% !split
\subsection{A structured programming approach}

\begin{itemize}
  \item Before writing a single line, have the algorithm clarified and understood. It is crucial to have a logical structure of e.g., the flow and organization of data before one starts writing.

  \item Always try to choose the simplest algorithm. Computational speed can be improved upon later.

  \item Try to write a as clear program as possible. Such programs are easier to debug, and although it may take more time, in the long run it may save you time. If you collaborate with other people, it reduces spending time on debuging and trying to understand what the codes do. A clear program will also allow you to remember better what the program really does!
\end{itemize}

\noindent
% !split
\subsection{A structured programming approach}

\begin{itemize}
  \item The planning of the program should be from top down to bottom, trying to keep the flow as linear as possible. Avoid jumping back and forth in the program. First you need to arrange the major tasks to be achieved. Then try to break the major tasks into subtasks. These can be represented by functions or subprograms. They should accomplish limited tasks and as far as possible be independent of each other. That will allow you to use them in other programs as well.

  \item Try always to find some cases where an analytical solution exists or where simple test cases can be applied. If possible, devise different algorithms for solving the same problem. If you get the same answers, you may have coded things correctly or made the same error twice or more.
\end{itemize}

\noindent

% ------------------- end of main content ---------------


% #ifdef PREAMBLE
\printindex

\end{document}
% #endif

