
% LaTeX Beamer file automatically generated from DocOnce
% https://github.com/hplgit/doconce

%-------------------- begin beamer-specific preamble ----------------------

\documentclass{beamer}

\usetheme{red_plain}
\usecolortheme{default}

% turn off the almost invisible, yet disturbing, navigation symbols:
\setbeamertemplate{navigation symbols}{}

% Examples on customization:
%\usecolortheme[named=RawSienna]{structure}
%\usetheme[height=7mm]{Rochester}
%\setbeamerfont{frametitle}{family=\rmfamily,shape=\itshape}
%\setbeamertemplate{items}[ball]
%\setbeamertemplate{blocks}[rounded][shadow=true]
%\useoutertheme{infolines}
%
%\usefonttheme{}
%\useinntertheme{}
%
%\setbeameroption{show notes}
%\setbeameroption{show notes on second screen=right}

% fine for B/W printing:
%\usecolortheme{seahorse}

\usepackage{pgf,pgfarrows,pgfnodes,pgfautomata,pgfheaps,pgfshade}
\usepackage{graphicx}
\usepackage{epsfig}
\usepackage{relsize}

\usepackage{fancybox}  % make sure fancybox is loaded before fancyvrb

\usepackage{fancyvrb}
%\usepackage{minted} % requires pygments and latex -shell-escape filename
%\usepackage{anslistings}
%\usepackage{listingsutf8}

\usepackage{amsmath,amssymb,bm}
%\usepackage[latin1]{inputenc}
\usepackage[T1]{fontenc}
\usepackage[utf8]{inputenc}
\usepackage{colortbl}
\usepackage[english]{babel}
\usepackage{tikz}
\usepackage{framed}
% Use some nice templates
\beamertemplatetransparentcovereddynamic

% --- begin table of contents based on sections ---
% Delete this, if you do not want the table of contents to pop up at
% the beginning of each section:
% (Only section headings can enter the table of contents in Beamer
% slides generated from DocOnce source, while subsections are used
% for the title in ordinary slides.)
\AtBeginSection[]
{
  \begin{frame}<beamer>[plain]
  \frametitle{}
  %\frametitle{Outline}
  \tableofcontents[currentsection]
  \end{frame}
}
% --- end table of contents based on sections ---

% If you wish to uncover everything in a step-wise fashion, uncomment
% the following command:

%\beamerdefaultoverlayspecification{<+->}

\newcommand{\shortinlinecomment}[3]{\note{\textbf{#1}: #2}}
\newcommand{\longinlinecomment}[3]{\shortinlinecomment{#1}{#2}{#3}}

\definecolor{linkcolor}{rgb}{0,0,0.4}
\hypersetup{
    colorlinks=true,
    linkcolor=linkcolor,
    urlcolor=linkcolor,
    pdfmenubar=true,
    pdftoolbar=true,
    bookmarksdepth=3
    }
\setlength{\parskip}{0pt}  % {1em}

\newenvironment{doconceexercise}{}{}
\newcounter{doconceexercisecounter}
\newenvironment{doconce:movie}{}{}
\newcounter{doconce:movie:counter}

\newcommand{\subex}[1]{\noindent\textbf{#1}}  % for subexercises: a), b), etc

%-------------------- end beamer-specific preamble ----------------------

% Add user's preamble




% insert custom LaTeX commands...

\raggedbottom
\makeindex

%-------------------- end preamble ----------------------

\begin{document}

% endif for #ifdef PREAMBLE



% ------------------- main content ----------------------



% ----------------- title -------------------------

\title{Slides from FYS3150/4150 Lectures, introduction to the course}

% ----------------- author(s) -------------------------

\author{Morten Hjorth-Jensen\inst{1,2}}
\institute{Department of Physics, University of Oslo\inst{1}
\and
Department of Physics and Astronomy and National Superconducting Cyclotron Laboratory, Michigan State University\inst{2}}
% ----------------- end author(s) -------------------------

\date{Fall 2015
% <optional titlepage figure>
}

\begin{frame}[plain,fragile]
\titlepage
\end{frame}

\begin{frame}[plain,fragile]
\frametitle{Overview of week 34}

\begin{block}{}
\begin{itemize}
  \item Thursday: First lecture: Presentation of the course, aims and content

  \item Thursday: Second Lecture: Introduction to C++ programming and numerical precision. Exercises for first week. 

  \item Friday: Numerical precision and C++ programming, continued and exercises for first week (chapter 2 of lecture notes)

  \item Numerical differentiation and loss of numerical precision (chapter 3 lecture notes)

  \item Computer lab: Thursday and Friday. First time: Thursday and Friday this week, Presentation of hardware and software at room FV329 first hour of every labgroup and solution of first simple exercises. The first two weeks we focus on simple programming exercises and to set up github and QTcreator.
\end{itemize}

\noindent
\end{block}
\end{frame}

\begin{frame}[plain,fragile]
\frametitle{Lectures and ComputerLab}

\begin{block}{}
\begin{itemize}
  \item Lectures: Thursday (8.15am-10am) and Friday (8.15am-10am).

  \item Weekly reading assignments needed to solve projects.

  \item First hour of each lab session may be used to discuss technicalities, address questions etc linked with projects.

  \item Detailed lecture notes, exercises, all programs presented, projects etc can be found at the homepage of the course.

  \item Computerlab: Thursday (10am-6pm) and Friday (10am-6pm) room FV329.

  \item Weekly plans and all other information are on the official webpage.

  \item Final written exam December 14, 2.30pm (four hours).
\end{itemize}

\noindent
\end{block}
\end{frame}

\begin{frame}[plain,fragile]
\frametitle{Course Format}

\begin{block}{}
\begin{itemize}
  \item Several computer exercises, 5 compulsory projects. Electronic reports only using \href{{https://devilry.ifi.uio.no/}}{devilry} to hand in projects and \href{{https://github.com/}}{Git} for repository and all your material.

  \item Evaluation and grading: The last two projects (4 and 5) count 1/3 of the final mark each  and the  final written exam counts another 1/3 of the final grade. Final written exam December 14.

  \item The computer lab (room FV329)consists of 16 Linux PCs, but many prefer own laptops. C/C++ is the default programming language, but Fortran2008 and Python are also used. All source codes discussed during the lectures can be found at the webpage and \href{{https://github.com/CompPhysics/ComputationalPhysics1}}{github address} of the course. We recommend either C/C++, Fortran2008 or Python as languages.
\end{itemize}

\noindent
\end{block}
\end{frame}

\begin{frame}[plain,fragile]
\frametitle{ComputerLab}

\begin{block}{}


{\footnotesize
\begin{tabular}{ll}
\hline
\multicolumn{1}{c}{ day } & \multicolumn{1}{c}{ teacher } \\
\hline
Thursday 10am-2pm & Anders, Morten L., Haavard, MHJ \\
Thursday 2pm-6pm  & Anders, Morten L., Haavard, MHJ \\
Friday 10am-2pm   & Anders, Morten L., Haavard, MHJ \\
Friday 2pm-6pm    & Anders, Morten L., Haavard, MHJ \\
\hline
\end{tabular}
}

\noindent
\end{block}
\end{frame}

\begin{frame}[plain,fragile]
\frametitle{Topics covered in this course}

\begin{block}{}
\begin{itemize}
  \item Numerical precision and intro to C++ programming

  \item Numerical derivation and integration

  \item Random numbers and Monte Carlo integration

  \item Monte Carlo methods in statistical physics

  \item Quantum Monte Carlo methods

  \item Linear algebra and eigenvalue problems

  \item Non-linear equations and roots of polynomials

  \item Ordinary differential equations

  \item Partial differential equations

  \item Parallelization of codes

  \item High-performance computing aspects
\end{itemize}

\noindent
\end{block}
\end{frame}

\begin{frame}[plain,fragile]
\frametitle{Syllabus FYS3150}

\begin{block}{Linear algebra and eigenvalue problems, chapters 6 and 7 }
\begin{itemize}
  \item Know Gaussian elimination and LU decomposition

  \item How to solve linear equations

  \item How to obtain the inverse and the determinant of a real symmetric matrix

  \item Cholesky and tridiagonal matrix decomposition
\end{itemize}

\noindent
\end{block}
\end{frame}

\begin{frame}[plain,fragile]
\frametitle{Syllabus FYS3150}

\begin{block}{Linear algebra and eigenvalue problems, chapters 6 and 7 }
\begin{itemize}
  \item Householder's tridiagonalization technique and finding eigenvalues based on this

  \item Jacobi's method for finding eigenvalues

  \item Singular value decomposition

  \item Qubic Spline interpolation
\end{itemize}

\noindent
\end{block}
\end{frame}

\begin{frame}[plain,fragile]
\frametitle{Syllabus FYS3150}

\begin{block}{Numerical integration, standard methods and Monte Carlo methods (chapters 4 and 11) }
\begin{itemize}
  \item Trapezoidal, rectangle and Simpson's rules

  \item Gaussian quadrature, emphasis on Legendre polynomials, but you need to know about other polynomials as well.

  \item Brute force Monte Carlo integration

  \item Random numbers (simplest algo, ran0) and probability distribution functions, expectation values

  \item Improved Monte Carlo integration and importance sampling.
\end{itemize}

\noindent
\end{block}
\end{frame}

\begin{frame}[plain,fragile]
\frametitle{Syllabus FYS3150}

\begin{block}{Monte Carlo methods in physics (chapters 12, 13, and 14) }
\begin{itemize}
  \item Random walks and Markov chains and relation with diffusion equation

  \item Metropolis algorithm, detailed balance and ergodicity

  \item Simple spin systems and phase transitions

  \item Variational Monte Carlo

  \item How to construct trial wave functions for quantum systems
\end{itemize}

\noindent
\end{block}
\end{frame}

\begin{frame}[plain,fragile]
\frametitle{Syllabus FYS3150}

\begin{block}{Ordinary differential equations (chapters 8 and 9) }
\begin{itemize}
  \item Euler's method and improved Euler's method, truncation errors

  \item Runge Kutta methods, 2nd and 4th order, truncation errors

  \item How to implement a second-order differential equation, both linear and non-linear. How to make your equations dimensionless.

  \item Boundary value problems, shooting and matching method (chap 9).
\end{itemize}

\noindent
\end{block}
\end{frame}

\begin{frame}[plain,fragile]
\frametitle{Syllabus FYS3150}

\begin{block}{Partial differential equations, chapter 10 }
\begin{itemize}
  \item Set up diffusion, Poisson and wave equations up to 2 spatial dimensions and time

  \item Set up the mathematical model and algorithms for these equations, with boundary and initial conditions. Their stability conditions.

  \item Explicit, implicit and Crank-Nicolson schemes, and how to solve them. Remember that they result in triangular matrices.

  \item How to compute the Laplacian in Poisson's equation.

  \item How to solve the wave equation in one and two dimensions.
\end{itemize}

\noindent
\end{block}
\end{frame}

\begin{frame}[plain,fragile]
\frametitle{Overarching aims of this course}

\begin{itemize}
  \item Develop a critical approach to all steps in a project, which methods are most relevant, which natural laws and physical processes are important. Sort out initial conditions and boundary conditions etc.

  \item This means to teach you structured scientific computing, learn to structure a project.

  \item A critical understanding of central mathematical algorithms and methods from numerical analysis. In particular their limits and stability criteria.

  \item Always try to find good checks of your codes (like solutions on closed form)

  \item To enable you to develop a critical view on the mathematical model and the physics.
\end{itemize}

\noindent
\end{frame}

\begin{frame}[plain,fragile]
\frametitle{And, there is nothing like a code which gives correct results!!}

% inline figure
\centerline{\includegraphics[width=0.6\linewidth]{fig-intro/Nebbdyr2.pdf}}



\begin{itemize}
 \item J. J. Barton and L. R. Nackman,*Scientific and Engineering C++*, Addison Wesley, 3rd edition 2000.

 \item B. Stoustrup, \emph{The C++ programming language}, Pearson, 1997.

 \item H. P. Langtangen INF-VERK3830 \href{{http://heim.ifi.uio.no/~hpl/INF-VERK4830/}}{\nolinkurl{http://heim.ifi.uio.no/~hpl/INF-VERK4830/}}

 \item D. Yang, \emph{C++ and Object-oriented Numeric Computing for Scientists and Engineers}, Springer 2000.
\end{itemize}

\noindent
\end{frame}

\begin{frame}[plain,fragile]
\frametitle{Other courses in Computational Science at UiO}

\begin{block}{Bachelor/Master/PhD Courses }
\begin{itemize}
  \item INF-MAT4350 Numerical linear algebra

  \item MAT-INF3300/3310, PDEs and Sobolev spaces I and II

  \item INF-MAT3360 PDEs

  \item INF5620 Numerical methods for PDEs, finite element method

  \item FYS4411 Computational physics II (Parallelization (MPI), object orientation, quantum mechanical systems with many interacting particles), spring semester

  \item FYS4460 Computational physics III (Parallelization (MPI), object orientation, classical statistical physics, simulation of phase transitions, spring semester

  \item INF3331 Problem solving with high-level languages (Python), fall semester

  \item INF3380 Parallel computing for problems in the Natural Sciences (mostly PDEs), spring semester
\end{itemize}

\noindent
\end{block}
\end{frame}

\begin{frame}[plain,fragile]
\frametitle{Extremely useful tools, strongly recommended}

\begin{block}{and discussed at the lab sessions }
\begin{itemize}
  \item GIT for version control (see webpage), this week

  \item ipython notebook, this week

  \item QTcreator for editing and mastering computational projects (for C++ codes, see webpage of course), next week

  \item Armadillo as a useful numerical library for C++, highly recommended, week 36 and rest of semester

  \item Unit tests, week 37 and later

  \item Devilry for handing in projects, next week
\end{itemize}

\noindent
\end{block}
\end{frame}

\end{document}
