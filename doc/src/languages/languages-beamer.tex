
% LaTeX Beamer file automatically generated from DocOnce
% https://github.com/hplgit/doconce

%-------------------- begin beamer-specific preamble ----------------------

\documentclass{beamer}

\usetheme{red_plain}
\usecolortheme{default}

% turn off the almost invisible, yet disturbing, navigation symbols:
\setbeamertemplate{navigation symbols}{}

% Examples on customization:
%\usecolortheme[named=RawSienna]{structure}
%\usetheme[height=7mm]{Rochester}
%\setbeamerfont{frametitle}{family=\rmfamily,shape=\itshape}
%\setbeamertemplate{items}[ball]
%\setbeamertemplate{blocks}[rounded][shadow=true]
%\useoutertheme{infolines}
%
%\usefonttheme{}
%\useinntertheme{}
%
%\setbeameroption{show notes}
%\setbeameroption{show notes on second screen=right}

% fine for B/W printing:
%\usecolortheme{seahorse}

\usepackage{pgf,pgfarrows,pgfnodes,pgfautomata,pgfheaps,pgfshade}
\usepackage{graphicx}
\usepackage{epsfig}
\usepackage{relsize}

\usepackage{fancyvrb}
\usepackage{minted} % requires pygments and latex -shell-escape filename
%\usepackage{anslistings}

\usepackage{amsmath,amssymb,bm}
%\usepackage[latin1]{inputenc}
\usepackage[utf8]{inputenc}
\usepackage{colortbl}
\usepackage[english]{babel}
\usepackage{tikz}
\usepackage{framed,anslistings}
% Use some nice templates
\beamertemplatetransparentcovereddynamic

% Delete this, if you do not want the table of contents to pop up at
% the beginning of each section:
\AtBeginSection[]
{
    \begin{frame}<beamer>[plain]
    \frametitle{}
    \tableofcontents[currentsection]
    \end{frame}
}

% Delete this, if you do not want the table of contents to pop up at
% the beginning of each section:
\AtBeginSection[]
{
    \begin{frame}<beamer>[plain]
    \frametitle{}
    \tableofcontents[currentsection]
    \end{frame}
}

% If you wish to uncover everything in a step-wise fashion, uncomment
% the following command:

%\beamerdefaultoverlayspecification{<+->}

\newcommand{\shortinlinecomment}[3]{\note{\textbf{#1}: #2}}
\newcommand{\longinlinecomment}[3]{\shortinlinecomment{#1}{#2}{#3}}

\definecolor{linkcolor}{rgb}{0,0,0.4}
\hypersetup{
    colorlinks=true,
    linkcolor=linkcolor,
    urlcolor=linkcolor,
    pdfmenubar=true,
    pdftoolbar=true,
    bookmarksdepth=3
    }
\setlength{\parskip}{7pt}  % {1em}
\newenvironment{doconceexercise}{}{}
\newcounter{doconceexercisecounter}
\newcommand{\subex}[1]{\noindent\textbf{#1}}  % for subexercises: a), b), etc

\newenvironment{notice_colors1admon}[1][]{\begin{block}{#1}}{\end{block}}
\newenvironment{notice_colors2admon}[1][]{\begin{block}{#1}}{\end{block}}
\newenvironment{notice_grayiconadmon}[1][]{\begin{block}{#1}}{\end{block}}
\newenvironment{notice_yellowiconadmon}[1][]{\begin{block}{#1}}{\end{block}}
\newenvironment{notice_mdfboxadmon}[1][]{\begin{block}{#1}}{\end{block}}
\newenvironment{summary_colors1admon}[1][]{\begin{block}{#1}}{\end{block}}
\newenvironment{summary_colors2admon}[1][]{\begin{block}{#1}}{\end{block}}
\newenvironment{summary_grayiconadmon}[1][]{\begin{block}{#1}}{\end{block}}
\newenvironment{summary_yellowiconadmon}[1][]{\begin{block}{#1}}{\end{block}}
\newenvironment{summary_mdfboxadmon}[1][]{\begin{block}{#1}}{\end{block}}
\newenvironment{warning_colors1admon}[1][]{\begin{block}{#1}}{\end{block}}
\newenvironment{warning_colors2admon}[1][]{\begin{block}{#1}}{\end{block}}
\newenvironment{warning_grayiconadmon}[1][]{\begin{block}{#1}}{\end{block}}
\newenvironment{warning_yellowiconadmon}[1][]{\begin{block}{#1}}{\end{block}}
\newenvironment{warning_mdfboxadmon}[1][]{\begin{block}{#1}}{\end{block}}
\newenvironment{question_colors1admon}[1][]{\begin{block}{#1}}{\end{block}}
\newenvironment{question_colors2admon}[1][]{\begin{block}{#1}}{\end{block}}
\newenvironment{question_grayiconadmon}[1][]{\begin{block}{#1}}{\end{block}}
\newenvironment{question_yellowiconadmon}[1][]{\begin{block}{#1}}{\end{block}}
\newenvironment{question_mdfboxadmon}[1][]{\begin{block}{#1}}{\end{block}}
\newenvironment{block_colors1admon}[1][]{\begin{block}{#1}}{\end{block}}
\newenvironment{block_colors2admon}[1][]{\begin{block}{#1}}{\end{block}}
\newenvironment{block_grayiconadmon}[1][]{\begin{block}{#1}}{\end{block}}
\newenvironment{block_yellowiconadmon}[1][]{\begin{block}{#1}}{\end{block}}
\newenvironment{block_mdfboxadmon}[1][]{\begin{block}{#1}}{\end{block}}
\newenvironment{paragraphadmon}[1][]{\begin{block}{#1}}{\end{block}}
\newenvironment{graybox2admon}[1][]{\begin{block}{#1}}{\end{block}}
\newcommand{\grayboxhrules}[1]{\begin{block}{}#1\end{block}}

\newenvironment{doconce:exercise}{}{}
\newcounter{doconce:exercise:counter}
\newenvironment{doconce:movie}{}{}
\newcounter{doconce:movie:counter}

%-------------------- end beamer-specific preamble ----------------------

% Add user's preamble




% insert custom LaTeX commands...

\raggedbottom
\makeindex

%-------------------- end preamble ----------------------

\begin{document}




% ------------------- main content ----------------------



% ----------------- title -------------------------

\title{Slides from FYS3150/4150 Programming aspects}

% ----------------- author(s) -------------------------

\author{Morten Hjorth-Jensen\inst{1,2}}
\institute{Department of Physics, University of Oslo\inst{1}
\and
Department of Physics and Astronomy and National Superconducting Cyclotron Laboratory, Michigan State University\inst{2}}
% ----------------- end author(s) -------------------------

\date{Fall 2015
% <optional titlepage figure>
}

\begin{frame}[plain,fragile]
\titlepage
\end{frame}

\begin{frame}[plain,fragile]
\frametitle{Extremely useful tools, strongly recommended}

\begin{block_mdfboxadmon}[and discussed at the lab sessions the first two weeks.]
\begin{itemize}
  \item GIT for version control (see webpage)

  \item ipython notebook

  \item QTcreator for editing and mastering computational projects (for C++ codes, see webpage of course)

  \item Armadillo as a useful numerical library for C++, highly recommended

  \item Unit tests, see also webpage

  \item Devilry for handing in projects
\end{itemize}

\noindent
\end{block_mdfboxadmon}
\end{frame}

\begin{frame}[plain,fragile]
\frametitle{A structured programming approach}

\begin{block_mdfboxadmon}[]

\begin{itemize}
  \item Before writing a single line, have the algorithm clarified and understood. It is crucial to have a logical structure of e.g., the flow and organization of data before one starts writing.

  \item Always try to choose the simplest algorithm. Computational speed can be improved upon later.

  \item Try to write a as clear program as possible. Such programs are easier to debug, and although it may take more time, in the long run it may save you time. If you collaborate with other people, it reduces spending time on debuging and trying to understand what the codes do. A clear program will also allow you to remember better what the program really does!
\end{itemize}

\noindent
\end{block_mdfboxadmon}
\end{frame}

\begin{frame}[plain,fragile]
\frametitle{A structured programming approach}

\begin{block_mdfboxadmon}[]

\begin{itemize}
  \item The planning of the program should be from top down to bottom, trying to keep the flow as linear as possible. Avoid jumping back and forth in the program. First you need to arrange the major tasks to be achieved. Then try to break the major tasks into subtasks. These can be represented by functions or subprograms. They should accomplish limited tasks and as far as possible be independent of each other. That will allow you to use them in other programs as well.

  \item Try always to find some cases where an analytical solution exists or where simple test cases can be applied. If possible, devise different algorithms for solving the same problem. If you get the same answers, you may have coded things correctly or made the same error twice or more.
\end{itemize}

\noindent
\end{block_mdfboxadmon}
\end{frame}

\begin{frame}[plain,fragile]
\frametitle{Getting Started}

\begin{block_mdfboxadmon}[Compiling and linking without QTcreator.]
In order to obtain an executable file for a C++ program, the following
instructions under Linux/Unix can be used

\begin{minted}[fontsize=\fontsize{9pt}{9pt},linenos=false,mathescape,baselinestretch=1.0,fontfamily=tt,xleftmargin=7mm]{console}
c++ -c -Wall myprogram.cpp
c++ -o myprogram myprogram.o
\end{minted}
where the compiler is called through the command c++/g++. The compiler
option -Wall means that a warning is issued in case of non-standard
language. The executable file is in this case \Verb!myprogram!. The option
\Verb!-c! is for compilation only, where the program is translated into machine code,
while the \Verb!-o! option links the produced object file \Verb!myprogram.o!
and produces the executable \Verb!myprogram! .

For Fortran2008 we use the Intel compiler, replace \Verb!c++! with \Verb!ifort!.
Also, to speed up the code use compile options like

\begin{minted}[fontsize=\fontsize{9pt}{9pt},linenos=false,mathescape,baselinestretch=1.0,fontfamily=tt,xleftmargin=7mm]{console}
c++ -O3 -c -Wall myprogram.cpp
\end{minted}
\end{block_mdfboxadmon}
\end{frame}

\begin{frame}[plain,fragile]
\frametitle{Makefiles and simple scripts}

Under Linux/Unix it is often convenient to create a
so-called makefile, which is a script which includes possible
compiling commands.

\begin{minted}[fontsize=\fontsize{9pt}{9pt},linenos=false,mathescape,baselinestretch=1.0,fontfamily=tt,xleftmargin=7mm]{c++}
# Comment lines
# General makefile for c - choose PROG =   name of given program
# Here we define compiler option, libraries and the  target
CC= g++ -Wall
PROG= myprogram
# this is the math library in C, not necessary for C++
LIB = -lm
# Here we make the executable file
${PROG} :          ${PROG}.o
                   ${CC} ${PROG}.o ${LIB} -o ${PROG}
# whereas here we create the object file
${PROG}.o :       ${PROG}.c
                  ${CC} -c ${PROG}.c
\end{minted}
If you name your file for \Verb!makefile!, simply type the command
\Verb!make! and Linux/Unix executes all of the statements in the above
makefile. Note that C++ files have the extension \Verb!.cpp!.
\end{frame}

\begin{frame}[plain,fragile]
\frametitle{Hello world}

\begin{block_mdfboxadmon}[The C encounter.]
Here we present first the C version.

\begin{minted}[fontsize=\fontsize{9pt}{9pt},linenos=false,mathescape,baselinestretch=1.0,fontfamily=tt,xleftmargin=7mm]{c++}
/* comments in C begin like this and end with */
#include <stdlib.h> /* atof function */
#include <math.h>   /* sine function */
#include <stdio.h>  /* printf function */
int main (int argc, char* argv[])
{
  double r, s;        /* declare variables */
  r = atof(argv[1]);  /* convert the text argv[1] to double */
  s = sin(r);
  printf("Hello, World! sin(%g)=%g\n", r, s);
  return 0;           /* success execution of the program */

\end{minted}
\end{block_mdfboxadmon}
\end{frame}

\begin{frame}[plain,fragile]
\frametitle{Hello World}

\begin{block_mdfboxadmon}[Dissection I.]
The compiler must see a declaration of a function before you can
call it (the compiler checks the argument and return types).
The declaration of library functions appears
in so-called ``header files'' that must be included in the program, e.g.,

\begin{minted}[fontsize=\fontsize{9pt}{9pt},linenos=false,mathescape,baselinestretch=1.0,fontfamily=tt,xleftmargin=7mm]{c++}
   #include <stdlib.h> /* atof function */
\end{minted}
We call three functions (atof, sin, printf)
and these are declared in three different header files.
The main program is a function called main
with a return value set to an integer, int (0 if success).
The operating system stores the return value,
and other programs/utilities can check whether
the execution was successful or not.
The command-line arguments are transferred to the main function through

\begin{minted}[fontsize=\fontsize{9pt}{9pt},linenos=false,mathescape,baselinestretch=1.0,fontfamily=tt,xleftmargin=7mm]{c++}
   int main (int argc, char* argv[])
\end{minted}
\end{block_mdfboxadmon}
\end{frame}

\begin{frame}[plain,fragile]
\frametitle{Hello World}

\begin{block_mdfboxadmon}[Dissection II.]
The command-line arguments are transferred to the main function through

\begin{minted}[fontsize=\fontsize{9pt}{9pt},linenos=false,mathescape,baselinestretch=1.0,fontfamily=tt,xleftmargin=7mm]{c++}
   int main (int argc, char* argv[])
\end{minted}
The integer \Verb!argc! is the no of command-line arguments, set to
one in our case, while
\Verb!argv! is a vector of strings containing the command-line arguments
with \Verb!argv[0]! containing  the name of the program
and \Verb!argv[1]!, \Verb!argv[2]!, ... are the command-line args, i.e., the number of
lines of input to the program.
Here we define floating points, see also below,
through the keywords \Verb!float! for single precision real numbers and
\Verb!double! for double precision. The function
\Verb!atof! transforms a text (\Verb!argv[1]!) to a float.
The sine function is declared in math.h, a library which
is not automatically included and needs to be linked when computing
an executable file.

With the command \Verb!printf! we obtain a formatted printout.
The \Verb!printf! syntax is used for formatting output
in many C-inspired languages (Perl, Python, Awk, partly C++).
\end{block_mdfboxadmon}
\end{frame}

\begin{frame}[plain,fragile]
\frametitle{Hello World}

\begin{block_mdfboxadmon}[Now in C++.]
Here we present first the C++ version.

\begin{minted}[fontsize=\fontsize{9pt}{9pt},linenos=false,mathescape,baselinestretch=1.0,fontfamily=tt,xleftmargin=7mm]{c++}
// A comment line begins like this in C++ programs
// Standard ANSI-C++ include files
using namespace std
#include <iostream>  // input and output
int main (int argc, char* argv[])
{
  // convert the text argv[1] to double using atof:
  double r = atof(argv[1]);
  double s = sin(r);
  cout << "Hello, World! sin(" << r << ")=" << s << '\n';
  // success
  return 0;
}
\end{minted}
\end{block_mdfboxadmon}
\end{frame}

\begin{frame}[plain,fragile]
\frametitle{C++ Hello World}

\begin{block_mdfboxadmon}[Dissection I.]
We have replaced the call to \Verb!printf! with the standard C++ function
\Verb!cout!. The header file \Verb!<iostream.h>! is then needed.
In addition, we don't need to
declare variables like \Verb!r! and \Verb!s!  at the beginning of the program.
I personally prefer
however to declare all variables at the beginning of a function, as this
gives \emph{me} a feeling of greater readability.
\end{block_mdfboxadmon}
\end{frame}

\begin{frame}[plain,fragile]
\frametitle{Brief summary}

\begin{block_mdfboxadmon}[C/C++ program.]
\begin{itemize}
  \item A C/C++ program begins with include statements of header files (libraries,intrinsic functions etc)

  \item Functions which are used are normally defined at top (details next week)

  \item The main program is set up as an integer, it returns 0 (everything correct) or 1 (something went wrong)

  \item Standard \Verb!if!, \Verb!while! and \Verb!for! statements as in Java, Fortran, Python...

  \item Integers have a very limited range.
\end{itemize}

\noindent
\end{block_mdfboxadmon}
\end{frame}

\begin{frame}[plain,fragile]
\frametitle{Brief summary}

\begin{block_mdfboxadmon}[Arrays.]
\begin{itemize}
  \item A C/C++ array begins by indexing at 0!

  \item Array allocations are done by size, not by the final index value.If you allocate an array with 10 elements, you should index them from $0,1,\dots, 9$.

  \item Initialize always an array before a computation.
\end{itemize}

\noindent
\end{block_mdfboxadmon}
\end{frame}

\begin{frame}[plain,fragile]
\frametitle{Serious problems and representation of numbers}

\begin{block_mdfboxadmon}[Integer and Real Numbers.]

\begin{itemize}
  \item Overflow

  \item Underflow

  \item Roundoff errors

  \item Loss of precision
\end{itemize}

\noindent
\end{block_mdfboxadmon}
\end{frame}

\begin{frame}[plain,fragile]
\frametitle{Limits, you must declare variables}

\begin{block_mdfboxadmon}[C++ and Fortran declarations.]


\begin{quote}
\begin{tabular}{lcl}
\hline
\multicolumn{1}{c}{ type in C/C++ and Fortran2008 } & \multicolumn{1}{c}{ bits } & \multicolumn{1}{c}{ range } \\
\hline
\Verb!int/INTEGER! (2)                         & 16                                             & -32768 to 32767                                \\
\Verb!unsigned int!                            & 16                                             & 0 to 65535                                     \\
\Verb!signed int!                              & 16                                             & -32768 to 32767                                \\
\Verb!short int!                               & 16                                             & -32768 to 32767                                \\
\Verb!unsigned short int!                      & 16                                             & 0 to 65535                                     \\
\Verb!signed short int!                        & 16                                             & $-32768$ to 32767                              \\
\Verb!int/long int/INTEGER! (4)                & 32                                             & -2147483648 to 2147483647                      \\
\Verb!signed long int!                         & 32                                             & -2147483648 to 2147483647                      \\
\Verb!float/REAL(4)!                           & 32                                             & $3.4\times 10^{-44}$ to $3.4\times 10^{+38}$   \\
\Verb!double/REAL(8)!                          & 64                                             & $1.7\times 10^{-322}$ to $1.7\times 10^{+308}$ \\
\Verb!long double!                             & 64                                             & $1.7\times 10^{-322}$ to $1.7\times 10^{+308}$ \\
\hline
\end{tabular}
\end{quote}

\noindent
\end{block_mdfboxadmon}
\end{frame}

\begin{frame}[plain,fragile]
\frametitle{From decimal to binary representation}

\begin{block_mdfboxadmon}[How to do it.]
\[
  a_n2^n+a_{n-1}2^{n-1}  +a_{n-2}2^{n-2}  +\dots +a_{0}2^{0}.
\]

In binary notation we have thus $(417)_{10} =(110110001)_2$
since we have

\[
(110100001)_2
=1\times2^8+1\times 2^{7}+0\times 2^{6}+1\times 2^{5}+0\times 2^{4}+0\times 2^{3}+0\times 2^{2}+0\times 2^{2}+0\times 2^{1}+1\times 2^{0}.
\]
To see this, we have performed the following divisions by 2




\begin{quote}
\begin{tabular}{ccc}
\hline
 \\
\hline
417/2=208                   & remainder 1                 & coefficient of $2^{0}$ is 1 \\
208/2=104                   & remainder 0                 & coefficient of $2^{1}$ is 0 \\
104/2=52                    & remainder 0                 & coefficient of $2^{2}$ is 0 \\
52/2=26                     & remainder 0                 & coefficient of $2^{3}$ is 0 \\
26/2=13                     & remainder 1                 & coefficient of $2^{4}$ is 0 \\
13/2= 6                     & remainder 1                 & coefficient of $2^{5}$ is 1 \\
6/2= 3                      & remainder 0                 & coefficient of $2^{6}$ is 0 \\
3/2= 1                      & remainder 1                 & coefficient of $2^{7}$ is 1 \\
1/2= 0                      & remainder 1                 & coefficient of $2^{8}$ is 1 \\
\hline
\end{tabular}
\end{quote}

\noindent
\end{block_mdfboxadmon}
\end{frame}

\begin{frame}[plain,fragile]
\frametitle{From decimal to binary representation}

\begin{block_mdfboxadmon}[Integer numbers.]
\begin{minted}[fontsize=\fontsize{9pt}{9pt},linenos=false,mathescape,baselinestretch=1.0,fontfamily=tt,xleftmargin=7mm]{c++}
using namespace std;
#include <iostream>
int main (int argc, char* argv[])
{
  int i;
  int terms[32]; // storage of a0, a1, etc, up to 32 bits
  int number = atoi(argv[1]);
  // initialise the term a0, a1 etc
  for (i=0; i < 32 ; i++){ terms[i] = 0;}
  for (i=0; i < 32 ; i++){
    terms[i] = number%2;
    number /= 2;

  // write out results
  cout << "Number of bytes used= " << sizeof(number) << endl;
  for (i=0; i < 32 ; i++){
    cout << " Term nr: " << i << "Value= " << terms[i];
    cout << endl;

  return 0;

\end{minted}
\end{block_mdfboxadmon}
\end{frame}

\begin{frame}[plain,fragile]
\frametitle{From decimal to binary representation}

\begin{block_mdfboxadmon}[Integer numbers Fortran.]
\begin{minted}[fontsize=\fontsize{9pt}{9pt},linenos=false,mathescape,baselinestretch=1.0,fontfamily=tt,xleftmargin=7mm]{fortran}
PROGRAM binary_integer
IMPLICIT NONE
  INTEGER  i, number, terms(0:31) ! storage of a0, a1, etc, up to 32 bits

  WRITE(*,*) 'Give a number to transform to binary notation'
  READ(*,*) number
! Initialise the terms a0, a1 etc
  terms = 0
! Fortran takes only integer loop variables
  DO i=0, 31
     terms(i) = MOD(number,2)
     number = number/2
  ENDDO
! write out results
  WRITE(*,*) 'Binary representation '
  DO i=0, 31
    WRITE(*,*)' Term nr and value', i, terms(i)
  ENDDO

END PROGRAM binary_integer
\end{minted}
\end{block_mdfboxadmon}
\end{frame}

\begin{frame}[plain,fragile]
\frametitle{Integer Numbers}

\begin{block_mdfboxadmon}[Possible Overflow for Integers.]
\begin{minted}[fontsize=\fontsize{9pt}{9pt},linenos=false,mathescape,baselinestretch=1.0,fontfamily=tt,xleftmargin=7mm]{c++}
// A comment line begins like this in C++ programs
// Program to calculate 2**n
// Standard ANSI-C++ include files */
using namespace std
#include <iostream>
#include <cmath>
int main()
{
   int  int1, int2, int3;
// print to screen
   cout << "Read in the exponential N for 2^N =\n";
// read from screen
   cin >> int2;
   int1 = (int) pow(2., (double) int2);
   cout << " 2^N * 2^N = " << int1*int1 << "\n";
   int3 = int1 - 1;
   cout << " 2^N*(2^N - 1) = " << int1 * int3  << "\n";
   cout << " 2^N- 1 = " << int3  << "\n";
   return 0;

// End: program main()
\end{minted}
\end{block_mdfboxadmon}
\end{frame}

\begin{frame}[plain,fragile]
\frametitle{Loss of Precision}

\begin{block_mdfboxadmon}[Machine Numbers.]
In the decimal system we would write a number like $9.90625$
in what is called the normalized scientific notation.

\[
  9.90625=0.990625\times 10^{1},
\]
and a real non-zero number could be generalized as

\begin{equation}
    x=\pm r\times 10^{{\mbox{n}}},
\end{equation}
with $r$ a number in the range $1/10 \le r < 1$.
In a similar way we can use represent a binary number in
scientific notation as

\begin{equation}
    x=\pm q\times 2^{{\mbox{m}}},
\end{equation}
with $q$ a number in the range $1/2 \le q < 1$.
This means that the mantissa of a binary number would be represented by
the general formula

\begin{equation}
(0.a_{-1}a_{-2}\dots a_{-n})_2=a_{-1}\times 2^{-1}
+a_{-2}\times 2^{-2}+\dots+a_{-n}\times 2^{-n}.
\end{equation}
\end{block_mdfboxadmon}
\end{frame}

\begin{frame}[plain,fragile]
\frametitle{Loss of Precision}

\begin{block_mdfboxadmon}[Machine Numbers.]
In a typical computer, floating-point numbers are represented
in the way described above, but with certain restrictions
on $q$ and $m$ imposed by the available word length.
In the machine, our
number $x$ is represented as

\begin{equation}
    x=(-1)^s\times {\mbox{mantissa}}\times 2^{{\mbox{exponent}}},
\end{equation}

where $s$ is the sign bit, and the exponent gives the available range.
With a single-precision word, 32 bits, 8 bits would typically be reserved
for the exponent,  1 bit for the sign and 23 for the mantissa.
\end{block_mdfboxadmon}
\end{frame}

\begin{frame}[plain,fragile]
\frametitle{Loss of Precision}

\begin{block_mdfboxadmon}[Machine Numbers.]
A modification of the scientific notation for binary numbers is to
require that the leading binary digit 1 appears to the left of the binary point.
In this case the representation of the mantissa $q$ would be
$(1.f)_2$ and $ 1 \le q < 2$. This form is rather useful when storing
binary numbers in a computer word, since we can always assume that the leading
bit 1 is there. One bit of space can then be saved meaning that a 23 bits
mantissa has actually 24 bits. This means explicitely that a binary number with 23 bits
for the mantissa reads

\begin{equation}
(1.a_{-1}a_{-2}\dots a_{-23})_2=1\times 2^0+a_{-1}\times 2^{-1}+
+a_{-2}\times 2^{-2}+\dots+a_{-23}\times 2^{-23}.
\end{equation}
As an example, consider the 32 bits binary number

\[
(10111110111101000000000000000000)_2,
\]
where the first bit is reserved for the sign, 1 in this case yielding a
negative sign. The exponent $m$ is given by the next 8 binary numbers
$01111101$ resulting in 125 in the decimal system.
\end{block_mdfboxadmon}
\end{frame}

\begin{frame}[plain,fragile]
\frametitle{Loss of Precision}

\begin{block_mdfboxadmon}[Machine Numbers.]
However, since the
exponent has eight bits, this means it has  $2^8-1=255$ possible numbers in the interval
$-128 \le m \le 127$, our final
exponent is $125-127=-2$ resulting in $2^{-2}$.
Inserting the sign and the mantissa yields the final number in the decimal representation as

\[
 -2^{-2}\left(1\times 2^0+1\times 2^{-1}+
1\times 2^{-2}+1\times 2^{-3}+0\times 2^{-4}+1\times 2^{-5}\right)=\]

\[
(-0.4765625)_{10}.
\]
In this case we have an exact machine representation with 32 bits (actually, we need less than
23 bits for the mantissa).

If our number $x$ can be exactly represented in the machine, we call
$x$ a machine number. Unfortunately, most numbers cannot  and are thereby
only approximated in the machine. When such a number occurs as the result
of reading some input data or of a computation, an inevitable error
will arise in representing it as accurately as possible by
a machine number.
\end{block_mdfboxadmon}
\end{frame}

\begin{frame}[plain,fragile]
\frametitle{Loss of Precision}

\begin{block_mdfboxadmon}[Machine Numbers.]
A floating number x, labelled $fl(x)$ will therefore always be represented as

\begin{equation}
  fl(x) = x(1\pm \epsilon_x),
\end{equation}
with $x$ the exact number and the error $|\epsilon_x| \le |\epsilon_M|$, where
$\epsilon_M$ is the precision assigned. A number like $1/10$ has no exact binary representation
with single or double precision. Since the mantissa

\[
\left(1.a_{-1}a_{-2}\dots a_{-n}\right)_2
\]
is always truncated at some stage $n$ due to its limited number of bits, there is only a
limited number of real binary numbers. The spacing between every real binary number is given by the
chosen machine precision.
For a 32 bit words this number is approximately
$ \epsilon_M \sim 10^{-7}$ and for double precision (64 bits) we have
$ \epsilon_M \sim 10^{-16}$, or in terms of a binary base
as $2^{-23}$ and $2^{-52}$ for single and double precision, respectively.
\end{block_mdfboxadmon}
\end{frame}

\begin{frame}[plain,fragile]
\frametitle{Loss of Precision}

\begin{block_mdfboxadmon}[Machine Numbers.]
In the machine a number is represented as

\begin{equation}
  fl(x)= x(1+\epsilon)
\end{equation}

where $|\epsilon| \leq \epsilon_M$ and $\epsilon$ is given by the
specified precision, $10^{-7}$ for single and $10^{-16}$ for double
precision, respectively.
$\epsilon_M$ is the given precision.
In case of a subtraction $a=b-c$, we have

\begin{equation}
   fl(a)=fl(b)-fl(c) = a(1+\epsilon_a),
\end{equation}
or

\begin{equation}
   fl(a)=b(1+\epsilon_b)-c(1+\epsilon_c),
\end{equation}

meaning that

\begin{equation}
   fl(a)/a=1+\epsilon_b\frac{b}{a}- \epsilon_c\frac{c}{a},
\end{equation}

and if $b\approx c$ we see that there is a potential for an increased
error in $fl(a)$.
\end{block_mdfboxadmon}
\end{frame}

\begin{frame}[plain,fragile]
\frametitle{Loss of Precision}

\begin{block_mdfboxadmon}[Machine Numbers.]
Define
the absolute error as

\begin{equation}
   |fl(a)-a|,
\end{equation}
whereas the relative error is

\begin{equation}
   \frac{ |fl(a)-a|}{a} \le \epsilon_a.
\end{equation}
The above subraction is thus

\begin{equation}
   \frac{ |fl(a)-a|}{a}=\frac{ |fl(b)-fl(c)-(b-c)|}{a},
\end{equation}
yielding

\begin{equation}
   \frac{ |fl(a)-a|}{a}=\frac{ |b\epsilon_b- c\epsilon_c|}{a}.
\end{equation}
The relative error
is the quantity of interest in scientific work. Information about the
absolute error is normally of little use in the absence of the magnitude
of the quantity being measured.
\end{block_mdfboxadmon}
\end{frame}

\begin{frame}[plain,fragile]
\frametitle{Loss of numerical precision}

Suppose we wish to evaluate the function

\[
   f(x)=\frac{1-\cos(x)}{\sin(x)},
\]
for small values of $x$. Five leading digits. If we multiply the denominator and numerator
with $1+\cos(x)$ we obtain the equivalent expression

\[
   f(x)=\frac{\sin(x)}{1+\cos(x)}.
\]

If we now choose $x=0.007$ (in radians) our choice of precision results in

\[
   \sin(0.007)\approx 0.69999\times 10^{-2},
\]
and

\[
   \cos(0.007)\approx 0.99998.
\]
\end{frame}

\begin{frame}[plain,fragile]
\frametitle{Loss of numerical precision}

The first expression for $f(x)$ results in

\[
   f(x)=\frac{1-0.99998}{0.69999\times 10^{-2}}=\frac{0.2\times 10^{-4}}{0.69999\times 10^{-2}}=0.28572\times 10^{-2},
\]
while the second expression results in

\[
   f(x)=\frac{0.69999\times 10^{-2}}{1+0.99998}=
\frac{0.69999\times 10^{-2}}{1.99998}=0.35000\times 10^{-2},
\]
which is also the exact result. In the first expression, due to our
choice of precision, we have
only one relevant digit in the numerator, after the
subtraction. This leads to a loss of precision and a wrong result due to
a cancellation of two nearly equal numbers.
If we had chosen a precision of six leading digits, both expressions
yield the same answer.
\end{frame}

\begin{frame}[plain,fragile]
\frametitle{Loss of numerical precision}

If we were to evaluate $x\sim \pi$, then the second expression for $f(x)$
can lead to potential losses of precision due to cancellations of nearly
equal numbers.

This simple example demonstrates  the loss of numerical precision due
to roundoff errors, where the number of leading digits is lost
in a subtraction of two near equal numbers.
The lesson to be drawn is that we cannot blindly compute a function.
We will always need to carefully analyze our algorithm in the search for
potential pitfalls. There is no magic recipe however, the only guideline
is an understanding of the fact that a machine cannot represent
correctly \emph{all} numbers.
\end{frame}

\begin{frame}[plain,fragile]
\frametitle{Loss of precision can cuae serious problems}

\begin{block_mdfboxadmon}[Real Numbers.]

\begin{itemize}
  \item \textbf{Overflow}: When the positive exponent exceeds the max value, e.g., 308 for \Verb!DOUBLE PRECISION! (64 bits). Under such circumstances the program will terminate and some compilers may give you the warning \Verb!OVERFLOW!.

  \item \textbf{Underflow}: When the negative exponent becomes smaller than the min value, e.g., -308 for \Verb!DOUBLE PRECISION!. Normally, the variable is then set to zero and the program continues. Other compilers (or compiler options) may warn you with the \Verb!UNDERFLOW! message and the program terminates.
\end{itemize}

\noindent
\end{block_mdfboxadmon}
\end{frame}

\begin{frame}[plain,fragile]
\frametitle{Loss of precision, real numbers}

\noindent\textbf{Roundoff errors.}
A floating point number like

\begin{equation}
   x= 1.234567891112131468 = 0.1234567891112131468\times 10^{1}
\end{equation}
may be stored in the following way. The exponent is  small
and is stored in full precision. However,
the mantissa is not stored fully. In double precision (64 bits), digits
beyond the
15th are lost since the mantissa is normally stored in two words,
one which is the most significant one representing
123456 and the least significant one containing 789111213. The digits
beyond 3 are lost. Clearly, if we are summing alternating series
with large numbers, subtractions between two large numbers may lead
to roundoff errors, since not all relevant digits are kept.
This leads eventually to the next problem, namely
\end{frame}

\begin{frame}[plain,fragile]
\frametitle{More on loss of precision}

\begin{block_mdfboxadmon}[Real Numbers.]

\begin{itemize}
  \item \textbf{Loss of precision}: When one has to e.g., multiply two large numbers where one suspects that the outcome may be beyond the bonds imposed by the variable declaration, one could represent the numbers by logarithms, or rewrite the equations to be solved in terms of dimensionless variables. When dealing with problems in e.g., particle physics or nuclear physics where distance is measured in fm ($10^{-15}$ m), it can be quite convenient to redefine the variables for distance in terms of a dimensionless variable of the order of unity. To give an example, suppose you work with single precision and wish to perform the addition $1+10^{-8}$. In this case, the information containing in $10^{-8}$ is simply lost in the addition. Typically, when performing the addition, the computer equates first the exponents of the two numbers to be added. For $10^{-8}$ this has however catastrophic consequences since in order to obtain an exponent equal to $10^0$, bits in the mantissa are shifted to the right. At the end, all bits in the mantissa are zeros.
\end{itemize}

\noindent
\end{block_mdfboxadmon}
\end{frame}

\begin{frame}[plain,fragile]
\frametitle{A problematic case}

\begin{block_mdfboxadmon}[Three ways of computing $e^{-x}$.]
Brute force:

\[\exp{(-x)}=\sum_{n=0}^{\infty}(-1)^n\frac{x^n}{n!}\]

Recursion relation for

\[ \exp{(-x)}=\sum_{n=0}^{\infty}s_n=\sum_{n=0}^{\infty}(-1)^n\frac{x^n}{n!} \]

\[ s_n=-s_{n-1}\frac{x}{n}, \]

\[ \exp{(x)}=\sum_{n=0}^{\infty}s_n \]

\[ \exp{(-x)}=\frac{1}{\exp{(x)}} \]
\end{block_mdfboxadmon}
\end{frame}

\begin{frame}[plain,fragile]
\frametitle{Program to compute $\exp{(-x)}$}

\begin{block_mdfboxadmon}[Brute Force.]
\begin{minted}[fontsize=\fontsize{9pt}{9pt},linenos=false,mathescape,baselinestretch=1.0,fontfamily=tt,xleftmargin=7mm]{c++}
// Program to calculate function exp(-x)
// using straightforward summation with differing  precision
using namespace std
#include <iostream>
#include <cmath>
// type float:  32 bits precision
// type double: 64 bits precision
#define   TYPE          double
#define   PHASE(a)      (1 - 2 * (abs(a) % 2))
#define   TRUNCATION    1.0E-10
// function declaration
TYPE factorial(int);
\end{minted}
\end{block_mdfboxadmon}
\end{frame}

\begin{frame}[plain,fragile]
\frametitle{Program to compute $\exp{(-x)}$}

\begin{block_mdfboxadmon}[Still Brute Force.]
\begin{minted}[fontsize=\fontsize{9pt}{9pt},linenos=false,mathescape,baselinestretch=1.0,fontfamily=tt,xleftmargin=7mm]{c++}
int main()
{
   int   n;
   TYPE  x, term, sum;
   for(x = 0.0; x < 100.0; x += 10.0)  {
     sum  = 0.0;                //initialization
     n    = 0;
     term = 1;
     while(fabs(term) > TRUNCATION)  {
         term =  PHASE(n) * (TYPE) pow((TYPE) x,(TYPE) n)
                / factorial(n);
         sum += term;
         n++;
     }  // end of while() loop
\end{minted}
\end{block_mdfboxadmon}
\end{frame}

\begin{frame}[plain,fragile]
\frametitle{Program to compute $\exp{(-x)}$}

\begin{block_mdfboxadmon}[Oh it never ends!]
\begin{minted}[fontsize=\fontsize{9pt}{9pt},linenos=false,mathescape,baselinestretch=1.0,fontfamily=tt,xleftmargin=7mm]{c++}
      printf("\nx = %4.1f   exp = %12.5E  series = %12.5E
              number of terms = %d",
              x, exp(-x), sum, n);
   } // end of for() loop

   printf("\n");           // a final line shift on output
   return 0;
} // End: function main()
//     The function factorial()
//     calculates and returns n!
TYPE factorial(int n)
{
   int  loop;
   TYPE fac;
   for(loop = 1, fac = 1.0; loop <= n; loop++)  {
      fac *= loop;

   return fac;
} // End: function factorial()
\end{minted}
\end{block_mdfboxadmon}
\end{frame}

\begin{frame}[plain,fragile]
\frametitle{Results $\exp{(-x)}$}

\begin{block_mdfboxadmon}[What is going on?]


\begin{quote}
\begin{tabular}{cccc}
\hline
\multicolumn{1}{c}{ $x$ } & \multicolumn{1}{c}{ $\exp{(-x)}$ } & \multicolumn{1}{c}{ Series } & \multicolumn{1}{c}{ Number of terms in series } \\
\hline
0.0                       & 0.100000E+01              & 0.100000E+01              & 1                         \\
10.0                      & 0.453999E-04              & 0.453999E-04              & 44                        \\
20.0                      & 0.206115E-08              & 0.487460E-08              & 72                        \\
30.0                      & 0.935762E-13              & -0.342134E-04             & 100                       \\
40.0                      & 0.424835E-17              & -0.221033E+01             & 127                       \\
50.0                      & 0.192875E-21              & -0.833851E+05             & 155                       \\
60.0                      & 0.875651E-26              & -0.850381E+09             & 171                       \\
70.0                      & 0.397545E-30              & NaN                       & 171                       \\
80.0                      & 0.180485E-34              & NaN                       & 171                       \\
90.0                      & 0.819401E-39              & NaN                       & 171                       \\
100.0                     & 0.372008E-43              & NaN                       & 171                       \\
\hline
\end{tabular}
\end{quote}

\noindent
\end{block_mdfboxadmon}
\end{frame}

\begin{frame}[plain,fragile]
\frametitle{Program to compute $\exp{(-x)}$}

\begin{block_mdfboxadmon}[]
\begin{minted}[fontsize=\fontsize{9pt}{9pt},linenos=false,mathescape,baselinestretch=1.0,fontfamily=tt,xleftmargin=7mm]{c++}
// program to compute exp(-x) without exponentials
using namespace std
#include <iostream>
#include <cmath>
#define  TRUNCATION     1.0E-10

int main()
{
   int       loop, n;
   double    x, term, sum;
   for(loop = 0; loop <= 100; loop += 10)
   {
     x    = (double) loop;          // initialization
     sum  = 1.0;
     term = 1;
     n    = 1;
\end{minted}
\end{block_mdfboxadmon}
\end{frame}

\begin{frame}[plain,fragile]
\frametitle{Program to compute $\exp{(-x)}$}

\begin{block_mdfboxadmon}[Last statements.]
\begin{minted}[fontsize=\fontsize{9pt}{9pt},linenos=false,mathescape,baselinestretch=1.0,fontfamily=tt,xleftmargin=7mm]{c++}
     while(fabs(term) > TRUNCATION)
       {
	 term *= -x/((double) n);
	 sum  += term;
	 n++;
       } // end while loop
     cout << "x = " << x   << " exp = " << exp(-x) <<"series = "
          << sum  << " number of terms =" << n << "\n";
   } // end of for() loop

   cout << "\n";           // a final line shift on output

}  /*    End: function main() */
\end{minted}
\end{block_mdfboxadmon}
\end{frame}

\begin{frame}[plain,fragile]
\frametitle{Results $\exp{(-x)}$}

\begin{block_mdfboxadmon}[More Problems.]


\begin{quote}
\begin{tabular}{cccc}
\hline
\multicolumn{1}{c}{ $x$ } & \multicolumn{1}{c}{ $\exp{(-x)}$ } & \multicolumn{1}{c}{ Series } & \multicolumn{1}{c}{ Number of terms in series } \\
\hline
0.000000                  & 0.10000000E+01            & 0.10000000E+01            & 1                         \\
10.000000                 & 0.45399900E-04            & 0.45399900E-04            & 44                        \\
20.000000                 & 0.20611536E-08            & 0.56385075E-08            & 72                        \\
30.000000                 & 0.93576230E-13            & -0.30668111E-04           & 100                       \\
40.000000                 & 0.42483543E-17            & -0.31657319E+01           & 127                       \\
50.000000                 & 0.19287498E-21            & 0.11072933E+05            & 155                       \\
60.000000                 & 0.87565108E-26            & -0.33516811E+09           & 182                       \\
70.000000                 & 0.39754497E-30            & -0.32979605E+14           & 209                       \\
80.000000                 & 0.18048514E-34            & 0.91805682E+17            & 237                       \\
90.000000                 & 0.81940126E-39            & -0.50516254E+22           & 264                       \\
100.000000                & 0.37200760E-43            & -0.29137556E+26           & 291                       \\
\hline
\end{tabular}
\end{quote}

\noindent
\end{block_mdfboxadmon}
\end{frame}

\begin{frame}[plain,fragile]
\frametitle{Most used formula for derivatives}

\begin{block_mdfboxadmon}[3 point formulae.]
First derivative  ($f_0 = f(x_0)$, $f_{-h}=f(x_0-h)$ and $f_{+h}=f(x_0+h)$

\[
   \frac{f_h-f_{-h}}{2h}=f'_0+\sum_{j=1}^{\infty}\frac{f_0^{(2j+1)}}{(2j+1)!}h^{2j}.
\]
Second derivative

\[
 \frac{ f_h -2f_0 +f_{-h}}{h^2}=f_0''+2\sum_{j=1}^{\infty}\frac{f_0^{(2j+2)}}{(2j+2)!}h^{2j}.
\]
\end{block_mdfboxadmon}
\end{frame}

\begin{frame}[plain,fragile]
\frametitle{Error Analysis}

\begin{block_mdfboxadmon}[]
\[
   \epsilon=log_{10}\left(\left|\frac{f''_{\mbox{computed}}-f''_{\mbox{exact}}}
                 {f''_{\mbox{exact}}}\right|\right),
\]

\[
   \epsilon_{\mbox{tot}}=\epsilon_{\mbox{approx}}+\epsilon_{\mbox{ro}}.
\]

For the computed second derivative  we have

\[
 f_0''=\frac{ f_h -2f_0 +f_{-h}}{h^2}-2\sum_{j=1}^{\infty}\frac{f_0^{(2j+2)}}{(2j+2)!}h^{2j},
\]
and the truncation or approximation error goes like

\[
  \epsilon_{\mbox{approx}}\approx \frac{f_0^{(4)}}{12}h^{2}.
\]
\end{block_mdfboxadmon}
\end{frame}

\begin{frame}[plain,fragile]
\frametitle{Error Analysis}

\begin{block_mdfboxadmon}[]
If we were not to worry about loss of precision, we could in principle
make $h$ as small as possible.
However, due to the computed expression in the above program example

\[
 f_0''=\frac{ f_h -2f_0 +f_{-h}}{h^2}=\frac{ (f_h -f_0) +(f_{-h}-f_0)}{h^2},
\]
we reach fairly quickly a limit for where loss of precision due to the subtraction
of two nearly equal numbers becomes crucial.

If $(f_{\pm h} -f_0)$ are very close, we have
$(f_{\pm h} -f_0)\approx \epsilon_M$, where $|\epsilon_M|\le 10^{-7}$ for single and
$|\epsilon_M|\le 10^{-15}$ for double precision, respectively.

We have then

\[
 \left|f_0''\right|=
 \left|\frac{ (f_h -f_0) +(f_{-h}-f_0)}{h^2}\right|\le \frac{ 2 \epsilon_M}{h^2}.
\]
\end{block_mdfboxadmon}
\end{frame}

\begin{frame}[plain,fragile]
\frametitle{Error Analysis}

\begin{block_mdfboxadmon}[]
Our total error becomes

\[
   \left|\epsilon_{\mbox{tot}}\right|\le  \frac{2 \epsilon_M}{h^2} +
                          \frac{f_0^{(4)}}{12}h^{2}.
    \label{eq:experror}
\]
It is then natural to ask which value of $h$ yields the smallest
total error. Taking the derivative of $\left|\epsilon_{\mbox{tot}}\right|$
with respect to $h$ results in

\[
   h= \left(\frac{ 24\epsilon_M}{f_0^{(4)}}\right)^{1/4}.
\]
With double precision and $x=10$ we obtain

\[
   h\approx 10^{-4}.
\]
Beyond this value, it is essentially the loss of numerical precision
which takes over.
\end{block_mdfboxadmon}
\end{frame}

\begin{frame}[plain,fragile]
\frametitle{Error Analysis}

\begin{block_mdfboxadmon}[]
Due to the subtractive cancellation in the expression
for $f''$ there is a pronounced detoriation in accuracy as $h$ is made smaller
and smaller.

It is instructive in this analysis to rewrite the numerator of
the computed derivative as

\[
   (f_h -f_0) +(f_{-h}-f_0)=(e^{x+h}-e^{x}) + (e^{x-h}-e^{x}),
\]
as

\[
   (f_h -f_0) +(f_{-h}-f_0)=e^x(e^{h}+e^{-h}-2),
\]
since it is the difference $(e^{h}+e^{-h}-2)$ which causes
the loss of precision.
\end{block_mdfboxadmon}
\end{frame}

\begin{frame}[plain,fragile]
\frametitle{Error Analysis}

\begin{block_mdfboxadmon}[]


\begin{quote}
\begin{tabular}{cccccc}
\hline
\multicolumn{1}{c}{ $x$ } & \multicolumn{1}{c}{ $h=0.01$ } & \multicolumn{1}{c}{ $h=0.001$ } & \multicolumn{1}{c}{ $h=0.0001$ } & \multicolumn{1}{c}{ $h=0.0000001$ } & \multicolumn{1}{c}{ Exact } \\
\hline
0.0           & 1.000008      & 1.000000      & 1.000000      & 1.010303      & 1.000000      \\
1.0           & 2.718304      & 2.718282      & 2.718282      & 2.753353      & 2.718282      \\
2.0           & 7.389118      & 7.389057      & 7.389056      & 7.283063      & 7.389056      \\
3.0           & 20.085704     & 20.085539     & 20.085537     & 20.250467     & 20.085537     \\
4.0           & 54.598605     & 54.598155     & 54.598151     & 54.711789     & 54.598150     \\
5.0           & 148.414396    & 148.413172    & 148.413161    & 150.635056    & 148.413159    \\
\hline
\end{tabular}
\end{quote}

\noindent
\end{block_mdfboxadmon}
\end{frame}

\begin{frame}[plain,fragile]
\frametitle{Error Analysis}

\begin{block_mdfboxadmon}[]
The results for $x=10$ are shown in the Table


\begin{quote}
\begin{tabular}{ccc}
\hline
\multicolumn{1}{c}{ $h$ } & \multicolumn{1}{c}{ $e^{h}+e^{-h}$ } & \multicolumn{1}{c}{ $e^{h}+e^{-h}-2$ } \\
\hline
$10^{-1}$                           & 2.0100083361116070                  & $1.0008336111607230\times 10^{-2}$  \\
$10^{-2}$                           & 2.0001000008333358                  & $1.0000083333605581\times 10^{-4}$  \\
$10^{-3}$                           & 2.0000010000000836                  & $1.0000000834065048\times 10^{-6}$  \\
$10^{-5}$                           & 2.0000000099999999                  & $1.0000000050247593\times 10^{-8}$  \\
$10^{-5}$                           & 2.0000000001000000                  & $9.9999897251734637\times 10^{-11}$ \\
$10^{-6}$                           & 2.0000000000010001                  & $9.9997787827987850\times 10^{-13}$ \\
$10^{-7}$                           & 2.0000000000000098                  & $9.9920072216264089\times 10^{-15}$ \\
$10^{-8}$                           & 2.0000000000000000                  & $0.0000000000000000\times 10^{0}$   \\
$10^{-9}$                           & 2.0000000000000000                  & $1.1102230246251565\times 10^{-16}$ \\
$10^{-10}$                          & 2.0000000000000000                  & $0.0000000000000000\times 10^{0}$   \\
\hline
\end{tabular}
\end{quote}

\noindent
\end{block_mdfboxadmon}




\noindent\textbf{\Large Week 35}
\end{frame}

\begin{frame}[plain,fragile]
\frametitle{Overview of week 35}

\begin{block_mdfboxadmon}[]
\begin{itemize}
  \item Monday: Repetition from last week

  \item Numerical differentiation

  \item C/C++ programming details, pointers, read/write to/from file

  \item Tuesday: Intro to linear Algebra and presentation of project 1.

  \item Matrices in C++ and Fortran2008

  \item Gaussian elimination and discussion of project 1.

  \item Computer-Lab: start project 1. Reading asssignments and preparation for project 1: sections 2.5 and 3.1 for general C++ and Fortran features. Sections 6.1-6.4 (till page 182) are relevant for project 1.
\end{itemize}

\noindent
\end{block_mdfboxadmon}
\end{frame}

\begin{frame}[plain,fragile]
\frametitle{Technical Matter in C/C++: Pointers}

\begin{block_mdfboxadmon}[]
A pointer specifies where a value resides in the computer's memory (like a house number specifies where a particular family resides on a street).

A pointer points to an address not to a data container of any kind!

Simple example declarations:

\begin{minted}[fontsize=\fontsize{9pt}{9pt},linenos=false,mathescape,baselinestretch=1.0,fontfamily=tt,xleftmargin=7mm]{c++}
  using namespace std; // note use of namespace
  int main()
 {
   // what are the differences?
   int var;
   cin >> var;
   int *p, q;
   int *s, *t;
   int * a new[var];    // dynamic memory allocation
   delete [] a;
}
\end{minted}
\end{block_mdfboxadmon}
\end{frame}

\begin{frame}[plain,fragile]
\frametitle{Technical Matter in C/C++: Pointer example I}

\begin{block_mdfboxadmon}[]
\begin{minted}[fontsize=\fontsize{9pt}{9pt},linenos=false,mathescape,baselinestretch=1.0,fontfamily=tt,xleftmargin=7mm]{c++}
using namespace std; // note use of namespace
int main()
{
  int var;
  int *p;
  p = &var;
  var  = 421;
  printf("Address of integer variable var : %p\n",&var);
  printf("Its value: %d\n", var);
  printf("Value of integer pointer p : %p\n",p);
  printf("The value p points at :  %d\n",*p);
  printf("Address of the pointer p : %p\n",&p);
  return 0;
}
\end{minted}
\end{block_mdfboxadmon}
\end{frame}

\begin{frame}[plain,fragile]
\frametitle{Dissection: Pointer example I}

\begin{block_mdfboxadmon}[Discussion.]

\begin{minted}[fontsize=\fontsize{9pt}{9pt},linenos=false,mathescape,baselinestretch=1.0,fontfamily=tt,xleftmargin=7mm]{c++}
int main()
{
  int var;     // Define an integer variable var
  int *p;      // Define a pointer to an integer
  p = &var;    // Extract the address of var
  var = 421;   // Change content of var
  printf("Address of integer variable var : %p\n", &var);
  printf("Its value: %d\n", var);  // 421
  printf("Value of integer pointer p : %p\n", p);  // = &var
  // The content of the variable pointed to by p is *p
  printf("The value p points at :  %d\n", *p);
  // Address where the pointer is stored in memory
  printf("Address of the pointer p : %p\n", &p);
  return 0;
}
\end{minted}
\end{block_mdfboxadmon}
\end{frame}

\begin{frame}[plain,fragile]
\frametitle{Pointer example II}

\begin{block_mdfboxadmon}[]
\begin{minted}[fontsize=\fontsize{9pt}{9pt},linenos=false,mathescape,baselinestretch=1.0,fontfamily=tt,xleftmargin=7mm]{c++}
int matr[2];
int *p;
p = &matr[0];
matr[0] = 321;
matr[1] = 322;
printf("\nAddress of matrix element matr[1]: %p",&matr[0]);
printf("\nValue of the  matrix element  matr[1]; %d",matr[0]);
printf("\nAddress of matrix element matr[2]: %p",&matr[1]);
printf("\nValue of the matrix element  matr[2]: %d\n", matr[1]);
printf("\nValue of the pointer p: %p",p);
printf("\nThe value p points to: %d",*p);
printf("\nThe value that (p+1) points to  %d\n",*(p+1));
printf("\nAddress of pointer p : %p\n",&p);
\end{minted}
\end{block_mdfboxadmon}
\end{frame}

\begin{frame}[plain,fragile]
\frametitle{Dissection: Pointer example II}

\begin{block_mdfboxadmon}[]
\begin{minted}[fontsize=\fontsize{9pt}{9pt},linenos=false,mathescape,baselinestretch=1.0,fontfamily=tt,xleftmargin=7mm]{c++}
int matr[2];    // Define integer array with two elements
int *p;         // Define pointer to integer
p = &matr[0];   // Point to the address of the first element in matr
matr[0] = 321;  // Change the first element
matr[1] = 322;  // Change the second element
printf("\nAddress of matrix element matr[1]: %p", &matr[0]);
printf("\nValue of the  matrix element  matr[1]; %d", matr[0]);
printf("\nAddress of matrix element matr[2]: %p", &matr[1]);
printf("\nValue of the matrix element  matr[2]: %d\n", matr[1]);
printf("\nValue of the pointer p: %p", p);
printf("\nThe value p points to: %d", *p);
printf("\nThe value that (p+1) points to  %d\n", *(p+1));
printf("\nAddress of pointer p : %p\n", &p);
\end{minted}
\end{block_mdfboxadmon}
\end{frame}

\begin{frame}[plain,fragile]
\frametitle{Output of Pointer example II}

\begin{block_mdfboxadmon}[]
\begin{Verbatim}[numbers=none,fontsize=\fontsize{9pt}{9pt},baselinestretch=0.95]
Address of the matrix element matr[1]: 0xbfffef70
Value of the  matrix element  matr[1]; 321
Address of the matrix element matr[2]: 0xbfffef74
Value of the matrix element  matr[2]: 322
Value of the pointer: 0xbfffef70
The value pointer points at: 321
The value that (pointer+1) points at:  322
Address of the pointer variable : 0xbfffef6c
\end{Verbatim}
\end{block_mdfboxadmon}
\end{frame}

\begin{frame}[plain,fragile]
\frametitle{File handling; C-way}

\begin{minted}[fontsize=\fontsize{9pt}{9pt},linenos=false,mathescape,baselinestretch=1.0,fontfamily=tt,xleftmargin=7mm]{c++}
using namespace std;
#include <iostream>
int main(int argc, char *argv[])
{
  FILE *in_file, *out_file;
  if( argc < 3)  {
    printf("The programs has the following structure :\n");
    printf("write in the name of the input and output files \n");
    exit(0);
  }
  in_file = fopen( argv[1], "r");// returns pointer to the  input file
  if( in_file == NULL )  { // NULL means that the file is missing
    printf("Can't find the input file %s\n", argv[1]);
    exit(0);
\end{minted}
\end{frame}

\begin{frame}[plain,fragile]
\frametitle{File handling; C way cont.}

\begin{minted}[fontsize=\fontsize{9pt}{9pt},linenos=false,mathescape,baselinestretch=1.0,fontfamily=tt,xleftmargin=7mm]{c++}
 out_file = fopen( argv[2], "w"); // returns a pointer to the output file
 if( out_file == NULL )  {       // can't find the file
    printf("Can't find the output file%s\n", argv[2]);
    exit(0);
  }
  fclose(in_file);
  fclose(out_file);
  return 0;
\end{minted}
\end{frame}

\begin{frame}[plain,fragile]
\frametitle{File handling, C++-way}

\begin{minted}[fontsize=\fontsize{9pt}{9pt},linenos=false,mathescape,baselinestretch=1.0,fontfamily=tt,xleftmargin=7mm]{c++}
#include <fstream>

// input and output file as global variable
ofstream ofile;
ifstream ifile;
\end{minted}
\end{frame}

\begin{frame}[plain,fragile]
\frametitle{File handling, C++-way}

\begin{minted}[fontsize=\fontsize{9pt}{9pt},linenos=false,mathescape,baselinestretch=1.0,fontfamily=tt,xleftmargin=7mm]{c++}
int main(int argc, char* argv[])
{
  char *outfilename;
  //Read in output file, abort if there are too
  //few command-line arguments
  if( argc <= 1 ){
    cout << "Bad Usage: " << argv[0] <<
      " read also output file on same line" << endl;
    exit(1);
  }
  else{
    outfilename=argv[1];
  }
  ofile.open(outfilename);
  .....
  ofile.close();  // close output file
\end{minted}
\end{frame}

\begin{frame}[plain,fragile]
\frametitle{File handling, C++-way}

\begin{minted}[fontsize=\fontsize{9pt}{9pt},linenos=false,mathescape,baselinestretch=1.0,fontfamily=tt,xleftmargin=7mm]{c++}
void output(double r_min , double r_max, int max_step,
            double *d)
{
int i;
ofile << "RESULTS:" << endl;
ofile << setiosflags(ios::showpoint | ios::uppercase);
ofile <<"R_min = " << setw(15) << setprecision(8) <<r_min <<endl;
ofile <<"R_max = " << setw(15) << setprecision(8) <<r_max <<endl;
ofile <<"Number of steps = " << setw(15) << max_step << endl;
ofile << "Five lowest eigenvalues:" << endl;
for(i = 0; i < 5; i++) {
    ofile << setw(15) << setprecision(8) << d[i] << endl;

}  // end of function output
\end{minted}
\end{frame}

\begin{frame}[plain,fragile]
\frametitle{File handling, C++-way}

\begin{minted}[fontsize=\fontsize{9pt}{9pt},linenos=false,mathescape,baselinestretch=1.0,fontfamily=tt,xleftmargin=7mm]{c++}
int main(int argc, char* argv[])
{
  char *infilename;
  // Read in input file, abort if there are too
  // few command-line arguments
  if( argc <= 1 ){
    cout << "Bad Usage: " << argv[0] <<
      " read also input file on same line" << endl;
    exit(1);
  }
  else{
    infilename=argv[1];
  }
  ifile.open(infilename);
  ....
  ifile.close();  // close input file
\end{minted}
\end{frame}

\begin{frame}[plain,fragile]
\frametitle{File handling, C++-way}

\begin{minted}[fontsize=\fontsize{9pt}{9pt},linenos=false,mathescape,baselinestretch=1.0,fontfamily=tt,xleftmargin=7mm]{c++}
const char* filename1 = "myfile";
ifstream ifile(filename1);
string filename2 = filename1 + ".out"
ofstream ofile(filename2);  // new output file
ofstream ofile(filename2, ios_base::app);  // append

//      Read something from the file:

double a; int b; char c[200];
ifile >> a >> b >> c;  // skips white space in between

//      Can test on success of reading:

if (!(ifile >> a >> b >> c)) ok = 0;
\end{minted}
\end{frame}

\begin{frame}[plain,fragile]
\frametitle{Call by value and reference}

\begin{minted}[fontsize=\fontsize{9pt}{9pt},linenos=false,mathescape,baselinestretch=1.0,fontfamily=tt,xleftmargin=7mm]{c++}
int main(int argc, char argv[]) {
int  a:
int *b;
a = 10;
b = new int[10];
for (i = 0; i < 10; i++) {
  b[i] = i;
}
func(a, b);
delete [] b;
return 0;
}
\end{minted}
\end{frame}

\begin{frame}[plain,fragile]
\frametitle{Call by value and reference}

Morten: Too complicated {\LaTeX} code for computer code to be
decoded....
\end{frame}

\begin{frame}[plain,fragile]
\frametitle{Call by value and reference}

\begin{itemize}
  \item Lines 1,2: Declaration of two variables \Verb!a! and \Verb!b!. The compiler reserves two locations in memory. The size of the location depends on the type of variable. Two properties are important for these locations: the address in memory and the content in the location. The value of \Verb!a! is \Verb!a!. The address of \Verb!a! is \Verb!&a!. The value of \Verb!b! is \Verb!*b!. The address of \Verb!b! is \Verb!&b!.

  \item Line 3: The value of \Verb!a! is now 10.

  \item Line 4: Memory to store 10 integers is reserved. The address to the first location is stored in b. Address to element number 6 is given by the expression (b + 6).

  \item Line 5: All 10 elements of \Verb!b! are given values: \Verb!b[0] = 0!, \Verb!b[1] = 1!, ....., \Verb!b[9] = 9!

  \item line 7: here we deallocate the variable \Verb!b!.
\end{itemize}

\noindent
\end{frame}

\begin{frame}[plain,fragile]
\frametitle{Call by value and reference}

\begin{itemize}
  \item Line 6: The \Verb!main()! function calls the function \Verb!func()! and the program counter transfers to the first statement in \Verb!func()!. With respect to data the following happens. The content of \Verb!a! (= 10) and the content of \Verb!b! (a memory address) are copied to a stack (new memory location) associated with the function \Verb!func()!

  \item Line 7: The variable \Verb!x! and \Verb!y! are local variables in \Verb!func()!. They have the values  \Verb!x = 10!, \Verb!y! is the address of the first element in \Verb!b! in \Verb!main()!.

  \item Line 8: The local variable \Verb!x! stored in the stack memory is changed to 17. Nothing happens with the value \Verb!a! in \Verb!main()!.
\end{itemize}

\noindent
\end{frame}

\begin{frame}[plain,fragile]
\frametitle{Call by value and reference}

\begin{itemize}
  \item Line 9: The value of y is an address and the symbol \Verb!*y! means the position in memory which has this address. The value in this location is now increased by 10. This means that the value of \Verb!b[0]! in the main program is equal to 10. Thus func() has modified a value in \Verb!main()!.

  \item Line 10: This statement has the same effect as line 9 except that it modifies the element \Verb!b[6]! in \Verb!main()! by adding a value of 10 to what was there originally, namely 5.

  \item Line 11: The program counter returns to \Verb!main()!, the next expression after \Verb!func(a,b)!. All data on the stack associated with \Verb!func()! are destroyed.
\end{itemize}

\noindent
\end{frame}

\begin{frame}[plain,fragile]
\frametitle{Call by value and reference}

\begin{itemize}
  \item The value of a is transferred to \Verb!func()! and stored in a new memory location called \Verb!x!. Any modification of \Verb!x! in \Verb!func()! does not affect in any way the value of \Verb!a! in \Verb!main()!. This is called \emph{transfer of data by value}.

  \item On the other hand the next argument in \Verb!func()! is an address which is transferred to \Verb!func()!. This address can be used to modify the corresponding value in \Verb!main()!. In the C language it is expressed as a modification of the value which \Verb!y! points to, namely the first element of \Verb!b!. This is called \emph{transfer of data by reference} and is a method to transfer data back to the calling function, in this case \Verb!main()!.
\end{itemize}

\noindent
\end{frame}

\begin{frame}[plain,fragile]
\frametitle{Call by value and reference}

C++ allows however the programmer to use solely call by reference
(note that call by reference is implemented as pointers).
To see the difference between C and C++, consider the following simple
examples. In C we would write

\begin{minted}[fontsize=\fontsize{9pt}{9pt},linenos=false,mathescape,baselinestretch=1.0,fontfamily=tt,xleftmargin=7mm]{c++}
   int n; n =8;
   func(&n); /* &n is a pointer to n */
   ....
   void func(int *i)
   {
     *i = 10; /* n is changed to 10 */
     ....
   }
\end{minted}

whereas in C++ we would write

\begin{minted}[fontsize=\fontsize{9pt}{9pt},linenos=false,mathescape,baselinestretch=1.0,fontfamily=tt,xleftmargin=7mm]{c++}
   int n; n =8;
   func(n); // just transfer n itself
   ....
   void func(int& i)
   {
     i = 10; // n is changed to 10
     ....
   }
\end{minted}
The reason why we emphasize the difference between call by value and call
by reference is that it allows the programmer to avoid pitfalls
like unwanted changes of variables. However, many people feel that this
reduces the readability of the code.
\end{frame}

\begin{frame}[plain,fragile]
\frametitle{Call by value and reference, F90/95}

In Fortran we can use \Verb!INTENT(IN)!, \Verb!INTENT(OUT)!, \Verb!INTENT(INOUT)! to let the
program know which values should or should not be changed.

\begin{minted}[fontsize=\fontsize{9pt}{9pt},linenos=false,mathescape,baselinestretch=1.0,fontfamily=tt,xleftmargin=7mm]{fortran}
SUBROUTINE coulomb_integral(np,lp,n,l,coulomb)
  USE effective_interaction_declar
  USE energy_variables
  USE wave_functions
  IMPLICIT NONE
  INTEGER, INTENT(IN)  :: n, l, np, lp
  INTEGER :: i
  REAL(KIND=8), INTENT(INOUT) :: coulomb
  REAL(KIND=8) :: z_rel, oscl_r, sum_coulomb
  ...
\end{minted}
This hinders unwanted changes and increases readability.
\end{frame}

\begin{frame}[plain,fragile]
\frametitle{Object orientation}

Why object orientation?

\begin{itemize}
  \item Three main topics: objects, class hierarchies and polymorphism

  \item The aim here is to be to be able to write a more general code which can easily be tailored to new situations.

  \item {\bf Polymorphism} is a term used in software development to describe a variety of techniques employed by programmers to create flexible and reusable software components. The term is Greek and it loosely translates to "many forms". Strategy: try to single out the variables needed to describe a given system and those needed to describe a given solver. !split
\end{itemize}

\noindent
\noindent\textbf{\large Object orientation}

In programming languages, a polymorphic object is an entity, such as a variable or a procedure, that can hold or operate on values of differing types during the program's execution. Because a polymorphic object can operate on a variety of values and types, it can also be used in a variety of programs, sometimes with little or no change by the programmer. The idea of write once, run many, also known as code reusability, is an important characteristic to the programming paradigm known as Object-Oriented Programming (OOP).

OOP describes an approach to programming where a program is viewed as a collection of interacting, but mostly independent software components. These software components are known as objects in OOP and they are typically implemented in a programming language as an entity that encapsulates both data and procedures.
\end{frame}

\begin{frame}[plain,fragile]
\frametitle{Programming classes}

In Fortran a vector or matrix start with $1$, but it is easy
to change a vector so that it starts with zero or even a negative number.
If we have a double precision Fortran vector  which starts at $-10$ and ends at $10$, we could declare it as
\Verb!REAL(KIND=8) ::  vector(-10:10)!. Similarly, if we want to start at zero and end at 10 we could write
\Verb!REAL(KIND=8) ::  vector(0:10)!.
We have also seen that Fortran  allows us to write a matrix addition ${\bf A} = {\bf B}+{\bf C}$ as
\Verb!A = B + C!.  This means that we have overloaded the addition operator so that it translates this operation into
two loops and an addition of two matrix elements $a_{ij} = b_{ij}+c_{ij}$.
\end{frame}

\begin{frame}[plain,fragile]
\frametitle{Programming classes}

The way the matrix addition is written is very close to the way we express this relation mathematically. The benefit for the
programmer is that our code is easier to read. Furthermore, such a way of coding makes it  more likely  to spot eventual
errors as well.

In Ansi C and C++ arrays start by default from $i=0$.  Moreover, if we  wish to add two matrices we need to explicitely write out
the two loops as

\begin{minted}[fontsize=\fontsize{9pt}{9pt},linenos=false,mathescape,baselinestretch=1.0,fontfamily=tt,xleftmargin=7mm]{c++}
   for(i=0 ; i < n ; i++) {
      for(j=0 ; j < n ; j++) {
         a[i][j]=b[i][j]+c[i][j]

\end{minted}
\end{frame}

\begin{frame}[plain,fragile]
\frametitle{Programming classes}

However,
the strength of C++ is the possibility
to define new data types, tailored to some particular problem.
Via new data types and overloading of operations such as addition and subtraction, we can easily define
sets of operations and data types which allow us to write a matrix addition in exactly the same
way as we would do in Fortran.  We could also change the way we declare a C++ matrix elements $a_{ij}$, from  $a[i][j]$
to say $a(i,j)$, as we would do in Fortran. Similarly, we could also change the default range from $0:n-1$ to $1:n$.

To achieve this we need to introduce two important entities in C++ programming, classes and templates.
\end{frame}

\begin{frame}[plain,fragile]
\frametitle{Programming classes}

The function and class declarations are fundamental concepts within C++.  Functions are abstractions
which encapsulate an algorithm or parts of it and perform specific tasks in a program.
We have already met several examples on how to use  functions.
Classes can be defined as abstractions which encapsulate
data and operations on these data.
The data can be very complex data structures  and the class can contain particular functions
which operate on these data. Classes allow therefore for a higher level of abstraction in computing.
The elements (or components) of the data
type are the class data members, and the procedures are the class
member functions.
\end{frame}

\begin{frame}[plain,fragile]
\frametitle{Programming classes}

Classes are user-defined tools used to create multi-purpose software which can be reused by other classes or functions.
These user-defined data types contain data (variables) and
functions operating on the data.

A simple example is that of a point in two dimensions.
The data could be the $x$ and $y$ coordinates of a given  point. The functions
we define could be simple read and write functions or the possibility to compute the distance between two points.
\end{frame}

\begin{frame}[plain,fragile]
\frametitle{Programming classes}

C++ has a class complex in its standard
template library (STL). The standard usage in a given function could then look like

\begin{minted}[fontsize=\fontsize{9pt}{9pt},linenos=false,mathescape,baselinestretch=1.0,fontfamily=tt,xleftmargin=7mm]{c++}
// Program to calculate addition and multiplication of two complex numbers
using namespace std;
#include <iostream>
#include <cmath>
#include <complex>
int main()
{
  complex<double> x(6.1,8.2), y(0.5,1.3);
  // write out x+y
  cout << x + y << x*y  << endl;
  return 0;

\end{minted}
where we add and multiply two complex numbers $x=6.1+\imath 8.2$ and $y=0.5+\imath 1.3$ with the obvious results
$z=x+y=6.6+\imath 9.5$ and $z=x\cdot y= -7.61+\imath 12.03$.
\end{frame}

\begin{frame}[plain,fragile]
\frametitle{Programming classes}

We proceed by  splitting our task in three files.

We define first a header file complex.h  which contains the declarations of
the class. The header file contains the class declaration (data and
functions), declaration of stand-alone functions, and all inlined
functions, starting as follows

\begin{minted}[fontsize=\fontsize{9pt}{9pt},linenos=false,mathescape,baselinestretch=1.0,fontfamily=tt,xleftmargin=7mm]{c++}
#ifndef Complex_H
#define Complex_H
//   various include statements and definitions
#include <iostream>          // Standard ANSI-C++ include files
#include <new>
#include ....

class Complex
{...
definition of variables and their character
};
//   declarations of various functions used by the class
...
#endif
\end{minted}
\end{frame}

\begin{frame}[plain,fragile]
\frametitle{Programming classes}

Next we provide a file complex.cpp where the code and algorithms of
different functions (except inlined functions) declared within the
class are written.  The files \Verb!complex.h! and \Verb!complex.cpp! are normally
placed in a directory with other classes and libraries we have
defined.

Finally, we discuss here an example of a main program which uses this
particular class.  An example of a program which uses our complex
class is given below. In particular we would like our class to perform
tasks like declaring complex variables, writing out the real and
imaginary part and performing algebraic operations such as adding or
multiplying two complex numbers.
\end{frame}

\begin{frame}[plain,fragile]
\frametitle{Programming classes}

\begin{minted}[fontsize=\fontsize{9pt}{9pt},linenos=false,mathescape,baselinestretch=1.0,fontfamily=tt,xleftmargin=7mm]{c++}
#include "Complex.h"
...  other include and declarations
int main ()
{
  Complex a(0.1,1.3);    // we declare a complex variable a
  Complex b(3.0), c(5.0,-2.3);  // we declare  complex variables b and c
  Complex d = b;         //  we declare  a new complex variable d
  cout << "d=" << d << ", a=" << a << ", b=" << b << endl;
  d = a*c + b/a;  //   we add, multiply and divide two complex numbers
  cout << "Re(d)=" << d.Re() << ", Im(d)=" << d.Im() << endl;  // write out of the real and imaginary parts

\end{minted}
\end{frame}

\begin{frame}[plain,fragile]
\frametitle{Programming classes}

We include the header file complex.h and define four different complex variables. These
are $a=0.1+\imath 1.3$, $b=3.0+\imath 0$ (note that if you don't define a value for the imaginary part  this is set to
zero), $c=5.0-\imath 2.3$ and $d=b$.  Thereafter we have defined standard algebraic operations and the member functions
of the class which allows us to print out the real and imaginary part of a given variable.
\end{frame}

\begin{frame}[plain,fragile]
\frametitle{Programming classes}

\begin{minted}[fontsize=\fontsize{9pt}{9pt},linenos=false,mathescape,baselinestretch=1.0,fontfamily=tt,xleftmargin=7mm]{c++}
class Complex
{
private:
   double re, im; // real and imaginary part
public:
   Complex ();                              // Complex c;
   Complex (double re, double im = 0.0); // Definition of a complex variable;
   Complex (const Complex& c);              // Usage: Complex c(a);   // equate two complex variables
   Complex& operator= (const Complex& c); // c = a;   //  equate two complex variables, same as previous
....

\end{minted}
\end{frame}

\begin{frame}[plain,fragile]
\frametitle{Programming classes}

\begin{minted}[fontsize=\fontsize{9pt}{9pt},linenos=false,mathescape,baselinestretch=1.0,fontfamily=tt,xleftmargin=7mm]{c++}
  ~Complex () {}                        // destructor
   double   Re () const;        // double real_part = a.Re();
   double   Im () const;        // double imag_part = a.Im();
   double   abs () const;       // double m = a.abs(); // modulus
   friend Complex operator+ (const Complex&  a, const Complex& b);
   friend Complex operator- (const Complex&  a, const Complex& b);
   friend Complex operator* (const Complex&  a, const Complex& b);
   friend Complex operator/ (const Complex&  a, const Complex& b);
};
\end{minted}
\end{frame}

\begin{frame}[plain,fragile]
\frametitle{Programming classes}

The class is defined via the statement \Verb!class Complex!. We must first use the key word
\Verb!class!, which in turn is followed by the user-defined variable name  \Verb!Complex!.
The body of the class, data and functions, is encapsulated  within the parentheses \Verb!{...}!.
\end{frame}

\begin{frame}[plain,fragile]
\frametitle{Programming classes}

Data and specific functions can be private, which means that they cannot be accessed from outside the class.
This means also that access cannot be inherited by other functions outside the class. If we use \Verb!protected!
instead of \Verb!private!, then data and functions can be inherited outside the class.
\end{frame}

\begin{frame}[plain,fragile]
\frametitle{Programming classes}

The key word \Verb!public! means  that data and functions can be accessed from outside the class.
Here we have defined several functions  which can be accessed by functions outside the class.
The declaration \Verb!friend! means that stand-alone functions can work on privately declared  variables  of the type
\Verb!(re, im)!.  Data members of a class should be declared as private variables.
\end{frame}

\begin{frame}[plain,fragile]
\frametitle{Programming classes}

The first public function we encounter is a so-called
constructor, which  tells how we declare a variable of type \Verb!Complex!
and how this variable is initialized. We have chose  three possibilities in the example above:

A declaration like \Verb!Complex c;! calls the member function \Verb!Complex()! which can have the following implementation

\begin{minted}[fontsize=\fontsize{9pt}{9pt},linenos=false,mathescape,baselinestretch=1.0,fontfamily=tt,xleftmargin=7mm]{c++}
Complex:: Complex () { re = im = 0.0; }
\end{minted}

meaning that it sets the real and imaginary parts to zero. Note the
way a member function is defined. The constructor is the first
function that is called when an object is instantiated.
\end{frame}

\begin{frame}[plain,fragile]
\frametitle{Programming classes}

Another possibility is

\begin{minted}[fontsize=\fontsize{9pt}{9pt},linenos=false,mathescape,baselinestretch=1.0,fontfamily=tt,xleftmargin=7mm]{c++}
Complex:: Complex () {}
\end{minted}
which means that there is no initialization of the real and imaginary parts. The drawback is that a given compiler can then assign random values to a given variable.

A call like \Verb!Complex a(0.1,1.3);! means that we could call the member function `Complex(double, double)`as

\begin{minted}[fontsize=\fontsize{9pt}{9pt},linenos=false,mathescape,baselinestretch=1.0,fontfamily=tt,xleftmargin=7mm]{c++}
Complex:: Complex (double re_a, double im_a) {
    re = re_a; im = im_a; }
\end{minted}
\end{frame}

\begin{frame}[plain,fragile]
\frametitle{Programming classes}

The simplest member function are those we defined to extract
the real and imaginary part of a variable. Here you have to recall that these are private data,
that is they invisible for users of the class.  We obtain a copy of these variables by defining the
functions

\begin{minted}[fontsize=\fontsize{9pt}{9pt},linenos=false,mathescape,baselinestretch=1.0,fontfamily=tt,xleftmargin=7mm]{c++}
double Complex:: Re () const { return re; }} //  getting the real part
double Complex:: Im () const { return im; }  //   and the imaginary part
\end{minted}
Note that we have introduced   the declaration  \Verb!const!.  What does it mean?
This declaration means that a variabale cannot be changed within  a called function.
\end{frame}

\begin{frame}[plain,fragile]
\frametitle{Programming classes}

If we define a variable as
\Verb!const double p = 3;! and then try to change its value, we will get an error when we
compile our program. This means that constant arguments in functions cannot be changed.

\begin{minted}[fontsize=\fontsize{9pt}{9pt},linenos=false,mathescape,baselinestretch=1.0,fontfamily=tt,xleftmargin=7mm]{c++}
// const arguments (in functions) cannot be changed:
void myfunc (const Complex& c)
{ c.re = 0.2; /* ILLEGAL!! compiler error... */  }
\end{minted}
If we declare the function and try to change the value to $0.2$, the compiler will complain by sending
an error message.
\end{frame}

\begin{frame}[plain,fragile]
\frametitle{Programming classes}

If we define a function to compute the absolute value of complex variable like

\begin{minted}[fontsize=\fontsize{9pt}{9pt},linenos=false,mathescape,baselinestretch=1.0,fontfamily=tt,xleftmargin=7mm]{c++}
double Complex:: abs ()  { return sqrt(re*re + im*im);}
\end{minted}
without the constant declaration  and define thereafter a function
\Verb!myabs! as

\begin{minted}[fontsize=\fontsize{9pt}{9pt},linenos=false,mathescape,baselinestretch=1.0,fontfamily=tt,xleftmargin=7mm]{c++}
double myabs (const Complex& c)
{ return c.abs(); }   // Not ok because c.abs() is not a const func.
\end{minted}
the compiler would not allow the c.abs() call in myabs
since \Verb!Complex::abs! is not a constant member function.
\end{frame}

\begin{frame}[plain,fragile]
\frametitle{Programming classes}

Constant functions cannot change the object's state.
To avoid this we declare the function \Verb!abs! as

\begin{minted}[fontsize=\fontsize{9pt}{9pt},linenos=false,mathescape,baselinestretch=1.0,fontfamily=tt,xleftmargin=7mm]{c++}
double Complex:: abs () const { return sqrt(re*re + im*im); }
\end{minted}
\end{frame}

\begin{frame}[plain,fragile]
\frametitle{Programming classes}

C++ (and Fortran) allow for overloading of operators. That means we
can define algebraic operations on for example vectors or any
arbitrary object.  As an example, a vector addition of the type ${\bf
c} = {\bf a} + {\bf b}$ means that we need to write a small part of
code with a for-loop over the dimension of the array.  We would rather
like to write this statement as \Verb!c = a+b;! as this makes the code much
more readable and close to eventual equations we want to code.  To
achieve this we need to extend the definition of operators.
\end{frame}

\begin{frame}[plain,fragile]
\frametitle{Programming classes}

Let us study the declarations in our complex class.
In our main function we have a statement like \Verb!d = b;!, which means
that we call \Verb!d.operator= (b)! and we have defined a so-called assignment operator
as a part of the class defined as

\begin{minted}[fontsize=\fontsize{9pt}{9pt},linenos=false,mathescape,baselinestretch=1.0,fontfamily=tt,xleftmargin=7mm]{c++}
Complex& Complex:: operator= (const Complex& c)
{
   re = c.re;
   im = c.im;
   return *this;
}
\end{minted}
\end{frame}

\begin{frame}[plain,fragile]
\frametitle{Programming classes}

With this function, statements like
\Verb!Complex d = b;! or \Verb!Complex d(b);!
make a new object $d$, which becomes a copy of $b$.
We can make simple implementations in terms of the assignment

\begin{minted}[fontsize=\fontsize{9pt}{9pt},linenos=false,mathescape,baselinestretch=1.0,fontfamily=tt,xleftmargin=7mm]{c++}
Complex:: Complex (const Complex& c)
{ *this = c; }
\end{minted}
which  is a pointer to "this object", \Verb!*this! is the present object,
so \Verb!*this = c;! means setting the present object equal to $c$, that is
\Verb!this->operator= (c);!.
\end{frame}

\begin{frame}[plain,fragile]
\frametitle{Programming classes}

The meaning of the addition operator $+$ for Complex objects is defined in the
function
\Verb!Complex operator+ (const Complex& a, const Complex& b); // a+b!
The compiler translates \Verb!c = a + b;! into \Verb!c = operator+ (a, b);!.
Since this implies the call to function, it brings in an additional overhead. If speed
is crucial and this function call is performed inside a loop, then it is more difficult for a
given compiler to perform optimizations of a loop.
\end{frame}

\begin{frame}[plain,fragile]
\frametitle{Programming classes}

The solution to this is to inline functions.   We discussed inlining in chapter
2 of the lecture notes.
Inlining means that the function body is copied directly into
the calling code, thus avoiding calling the function.
Inlining is enabled by the inline keyword

\begin{minted}[fontsize=\fontsize{9pt}{9pt},linenos=false,mathescape,baselinestretch=1.0,fontfamily=tt,xleftmargin=7mm]{c++}
inline Complex operator+ (const Complex& a, const Complex& b)
{ return Complex (a.re + b.re, a.im + b.im); }
\end{minted}
Inline functions, with complete bodies must be written in the header file  complex.h.
\end{frame}

\begin{frame}[plain,fragile]
\frametitle{Programming classes}

Consider  the case \Verb!c = a + b;!
that is,  \Verb!c.operator= (operator+ (a,b));!
If \Verb!operator+!, \Verb!operator=! and the constructor \Verb!Complex(r,i)! all
are inline functions, this transforms to

\begin{minted}[fontsize=\fontsize{9pt}{9pt},linenos=false,mathescape,baselinestretch=1.0,fontfamily=tt,xleftmargin=7mm]{c++}
c.re = a.re + b.re;
c.im = a.im + b.im;
\end{minted}
by the compiler, i.e., no function calls
\end{frame}

\begin{frame}[plain,fragile]
\frametitle{Programming classes}

The stand-alone function \Verb!operator+! is a friend of the Complex  class

\begin{minted}[fontsize=\fontsize{9pt}{9pt},linenos=false,mathescape,baselinestretch=1.0,fontfamily=tt,xleftmargin=7mm]{c++}
class Complex
{
   ...
   friend Complex operator+ (const Complex& a, const Complex& b);
   ...
};
\end{minted}
so it can read (and manipulate) the private data parts $re$ and
$im$ via

\begin{minted}[fontsize=\fontsize{9pt}{9pt},linenos=false,mathescape,baselinestretch=1.0,fontfamily=tt,xleftmargin=7mm]{c++}
inline Complex operator+ (const Complex& a, const Complex& b)
{ return Complex (a.re + b.re, a.im + b.im); }
\end{minted}
\end{frame}

\begin{frame}[plain,fragile]
\frametitle{Programming classes}

Since we do not need to alter the re and im variables, we can
get the values by Re() and Im(), and there is no need to be a
friend function

\begin{minted}[fontsize=\fontsize{9pt}{9pt},linenos=false,mathescape,baselinestretch=1.0,fontfamily=tt,xleftmargin=7mm]{c++}
inline Complex operator+ (const Complex& a, const Complex& b)
{ return Complex (a.Re() + b.Re(), a.Im() + b.Im()); }
\end{minted}
\end{frame}

\begin{frame}[plain,fragile]
\frametitle{Programming classes}

The multiplication functionality can now be extended to imaginary numbers by the following code

\begin{minted}[fontsize=\fontsize{9pt}{9pt},linenos=false,mathescape,baselinestretch=1.0,fontfamily=tt,xleftmargin=7mm]{c++}
inline Complex operator* (const Complex& a, const Complex& b)
{
  return Complex(a.re*b.re - a.im*b.im, a.im*b.re + a.re*b.im);

\end{minted}
It will be convenient to inline all functions used by this operator.
\end{frame}

\begin{frame}[plain,fragile]
\frametitle{Programming classes}

To inline the complete expression \Verb!a*b;!, the constructors and
\Verb!operator=!  must also be inlined.  This can be achieved via the following piece of code

\begin{minted}[fontsize=\fontsize{9pt}{9pt},linenos=false,mathescape,baselinestretch=1.0,fontfamily=tt,xleftmargin=7mm]{c++}
inline Complex:: Complex () { re = im = 0.0; }
inline Complex:: Complex (double re_, double im_)
{ ... }
inline Complex:: Complex (const Complex& c)
{ ... }
inline Complex:: operator= (const Complex& c)
{ ... }
\end{minted}
\end{frame}

\begin{frame}[plain,fragile]
\frametitle{Programming classes}

\begin{minted}[fontsize=\fontsize{9pt}{9pt},linenos=false,mathescape,baselinestretch=1.0,fontfamily=tt,xleftmargin=7mm]{c++}
// e, c, d are complex
e = c*d;
// first compiler translation:
e.operator= (operator* (c,d));
// result of nested inline functions
// operator=, operator*, Complex(double,double=0):
e.re = c.re*d.re - c.im*d.im;
e.im = c.im*d.re + c.re*d.im;
\end{minted}
The definitions \Verb!operator-! and \Verb!operator/! follow the same set up.
\end{frame}

\begin{frame}[plain,fragile]
\frametitle{Programming classes}

Finally, if we wish to write to file or another device a complex number using the simple syntax
\Verb!cout << c;!, we obtain this by defining
the effect of $<<$ for a Complex object as

\begin{minted}[fontsize=\fontsize{9pt}{9pt},linenos=false,mathescape,baselinestretch=1.0,fontfamily=tt,xleftmargin=7mm]{c++}
ostream& operator<< (ostream& o, const Complex& c)
{ o << "(" << c.Re() << "," << c.Im() << ") "; return o;}
\end{minted}
\end{frame}

\begin{frame}[plain,fragile]
\frametitle{Programming classes, templates}

What if we wanted to make a class which takes integers
or floating point numbers with single precision?
A simple way to achieve this is copy and paste our class and replace \Verb!double! with for
example \Verb!int!.

C++  allows us to do this automatically via the usage of templates, which
are the C++ constructs for parameterizing parts of
classes. Class templates  is a template for producing classes. The declaration consists
of the keyword \Verb!template! followed by a list of template arguments enclosed in brackets.
\end{frame}

\begin{frame}[plain,fragile]
\frametitle{Programming classes}

We can therefore make a more general class by rewriting our original example as

\begin{minted}[fontsize=\fontsize{9pt}{9pt},linenos=false,mathescape,baselinestretch=1.0,fontfamily=tt,xleftmargin=7mm]{c++}
template<class T>
class Complex
{
private:
   T re, im; // real and imaginary part
public:
   Complex ();                              // Complex c;
   Complex (T re, T im = 0); // Definition of a complex variable;
   Complex (const Complex& c);              // Usage: Complex c(a);   // equate two complex variables
   Complex& operator= (const Complex& c); // c = a;   //  equate two complex variables, same as previous

\end{minted}
\end{frame}

\begin{frame}[plain,fragile]
\frametitle{Programming classes}

We can therefore make a more general class by rewriting our original example as

\begin{minted}[fontsize=\fontsize{9pt}{9pt},linenos=false,mathescape,baselinestretch=1.0,fontfamily=tt,xleftmargin=7mm]{c++}
  ~Complex () {}                        // destructor
   T   Re () const;        // T real_part = a.Re();
   T   Im () const;        // T imag_part = a.Im();
   T   abs () const;       // T m = a.abs(); // modulus
   friend Complex operator+ (const Complex&  a, const Complex& b);
   friend Complex operator- (const Complex&  a, const Complex& b);
   friend Complex operator* (const Complex&  a, const Complex& b);
   friend Complex operator/ (const Complex&  a, const Complex& b);
};
\end{minted}
\end{frame}

\begin{frame}[plain,fragile]
\frametitle{Programming classes}

What it says is that \Verb!Complex! is a parameterized type with $T$ as a parameter and $T$ has to be a type such as double
or float.
The class complex is now a class template
and we would define variables in a code as

\begin{minted}[fontsize=\fontsize{9pt}{9pt},linenos=false,mathescape,baselinestretch=1.0,fontfamily=tt,xleftmargin=7mm]{c++}
Complex<double> a(10.0,5.1);
Complex<int> b(1,0);
\end{minted}
\end{frame}

\begin{frame}[plain,fragile]
\frametitle{Programming classes}

Member functions of our class are defined by preceding the name of the function with the \Verb!template! keyword.
Consider the function we defined as \Verb!Complex:: Complex (double re_a, double im_a)!.
We would rewrite this function as

\begin{minted}[fontsize=\fontsize{9pt}{9pt},linenos=false,mathescape,baselinestretch=1.0,fontfamily=tt,xleftmargin=7mm]{c++}
template<class T>
Complex<T>:: Complex (T re_a, T im_a)
{ re = re_a; im = im_a; }
\end{minted}
The member functions  are otherwise defined following ordinary member function definitions.
\end{frame}

\begin{frame}[plain,fragile]
\frametitle{Programming classes}

Here follows a very simple first class in the file squared.h

\begin{minted}[fontsize=\fontsize{9pt}{9pt},linenos=false,mathescape,baselinestretch=1.0,fontfamily=tt,xleftmargin=7mm]{c++}
// Not all declarations here
// Class to compute the square of a number
template<class T>
class Squared{
  public:
    // Default constructor, not used here
    Squared(){}

    // Overload the function operator()
    T operator()(T x){return x*x;}

};
\end{minted}
\end{frame}

\begin{frame}[plain,fragile]
\frametitle{Programming classes}

and we would use it as

\begin{minted}[fontsize=\fontsize{9pt}{9pt},linenos=false,mathescape,baselinestretch=1.0,fontfamily=tt,xleftmargin=7mm]{c++}
#include <iostream>
#include "squared.h"
using namespace std;

int main(){
  Squared<double> s;
  cout << s(3) << endl;

\end{minted}
\end{frame}

\begin{frame}[plain,fragile]
\frametitle{Unit Testing}

\begin{block_mdfboxadmon}[]
Unit Testing is the practice of testing the smallest testable parts,
called units, of an application individually and independently to
determine if they behave exactly as expected. Unit tests (short code
fragments) are usually written such that they can be preformed at any
time during the development to continually verify the behavior of the
code. In this way, possible bugs will be identified early in the
development cycle, making the debugging at later stage much
easier. There are many benefits associated with Unit Testing, such as
\begin{itemize}
  \item It increases confidence in changing and maintaining code. Big changes can be made to the code quickly, since the tests will ensure that everything still is working properly.

  \item Since the code needs to be modular to make Unit Testing possible, the code will be easier to reuse. This improves the code design.

  \item Debugging is easier, since when a test fails, only the latest changes need to be debugged.
\begin{itemize}

   \item Different parts of a project can be tested without the need to wait for the other parts to be available.

\end{itemize}

\noindent
  \item A unit test can serve as a documentation on the functionality of a unit of the code.
\end{itemize}

\noindent
\end{block_mdfboxadmon}
\end{frame}

\begin{frame}[plain,fragile]
\frametitle{Simple example of unit test}

\begin{block_mdfboxadmon}[]
Look up the guide on how to install unit tests for c++ at course webpage. This is the version with classes.
\begin{minted}[fontsize=\fontsize{9pt}{9pt},linenos=false,mathescape,baselinestretch=1.0,fontfamily=tt,xleftmargin=7mm]{c++}
#include <unittest++/UnitTest++.h>

class MyMultiplyClass{
public:
    double multiply(double x, double y) {
        return x * y;
    }
};

TEST(MyMath) {
    MyMultiplyClass my;
    CHECK_EQUAL(56, my.multiply(7,8));
}

int main()
{
    return UnitTest::RunAllTests();
}
\end{minted}
\end{block_mdfboxadmon}
\end{frame}

\begin{frame}[plain,fragile]
\frametitle{Simple example of unit test}

\begin{block_mdfboxadmon}[]
And without classes
\begin{minted}[fontsize=\fontsize{9pt}{9pt},linenos=false,mathescape,baselinestretch=1.0,fontfamily=tt,xleftmargin=7mm]{c++}
#include <unittest++/UnitTest++.h>


double multiply(double x, double y) {
    return x * y;
}

TEST(MyMath) {
    CHECK_EQUAL(56, multiply(7,8));
}

int main()
{
    return UnitTest::RunAllTests();
} 
\end{minted}
For Fortran users, the link at \href{{http://sourceforge.net/projects/fortranxunit/}}{\nolinkurl{http://sourceforge.net/projects/fortranxunit/}} contains a similar
software for unit testing.
\end{block_mdfboxadmon}





\noindent\textbf{\large The report: how to write a good scienfitic/technical report}

\begin{block_mdfboxadmon}[What should it contain? A typical structure.]
\begin{itemize}
 \item An introduction where you explain the aims and rationale for the physics case and  what you have done. At the end of the introduction you should give a brief summary of the structure of the report

 \item Theoretical models and technicalities. This is the methods section.

 \item Results and discussion

 \item Conclusions and perspectives

 \item Appendix with extra material

 \item Bibliography
\end{itemize}

\noindent
Keep always a good log of what you do.
\end{block_mdfboxadmon}




\noindent\textbf{\large The report}

\begin{block_mdfboxadmon}[What should I focus on? Introduction.]
You don't need to answer all questions in a chronological order.  When you write the introduction you could focus on the following aspects
\begin{itemize}
 \item Motivate the reader, the first part of the introduction gives always a motivation and tries to give the overarching ideas

 \item What I have done

 \item The structure of the report, how it is organized etc
\end{itemize}

\noindent
\end{block_mdfboxadmon}
\end{frame}

\begin{frame}[plain,fragile]
\frametitle{The report}

\begin{block_mdfboxadmon}[What should I focus on? Methods sections.]
\begin{itemize}
 \item Describe the methods and algorithms

 \item You need to explain how you implemented the methods and also say something about the structure of your algorithm and present some parts of your code

 \item You should plug in some calculations to demonstrate your code, such as selected runs used to validate and verify your results. The latter is extremely important!!  A reader needs to understand that your code reproduces selected benchmarks and reproduces previous results, either numerical and/or well-known  closed form expressions.
\end{itemize}

\noindent
\end{block_mdfboxadmon}
\end{frame}

\begin{frame}[plain,fragile]
\frametitle{The report}

\begin{block_mdfboxadmon}[What should I focus on? Results.]
\begin{itemize}
 \item Present your results

 \item Give a critical discussion of your work and place it in the correct context.

 \item Relate your work to other calculations/studies

 \item An eventual reader should be able to reproduce your calculations if she/he wants to do so. All input variables should be properly explained.

 \item Make sure that figures and tables should contain enough information in their captions, axis labels etc so that an eventual reader can gain a first impression of your work by studying figures and tables only.
\end{itemize}

\noindent
\end{block_mdfboxadmon}
\end{frame}

\begin{frame}[plain,fragile]
\frametitle{The report}

\begin{block_mdfboxadmon}[What should I focus on? Conclusions.]
\begin{itemize}
 \item State your main findings and interpretations

 \item Try as far as possible to present perspectives for future work

 \item Try to discuss the pros and cons of the methods and possible improvements
\end{itemize}

\noindent
\end{block_mdfboxadmon}
\end{frame}

\begin{frame}[plain,fragile]
\frametitle{The report}

\begin{block_mdfboxadmon}[What should I focus on? additional material.]
\begin{itemize}
 \item Additional calculations used to validate the codes

 \item Selected calculations, these can be listed with  few comments

 \item Listing of the code if you feel this is necessary
\end{itemize}

\noindent
You can consider moving parts of the material from the methods section to the appendix. You can also place additional material on your webpage.
\end{block_mdfboxadmon}
\end{frame}

\begin{frame}[plain,fragile]
\frametitle{The report}

\begin{block_mdfboxadmon}[What should I focus on? References.]
\begin{itemize}
 \item Give always references to material you base your work on, either  scientific articles/reports or books.

 \item Refer to articles as: name(s) of author(s), journal, volume (boldfaced), page and year in parenthesis.

 \item Refer to books as: name(s) of author(s), title of book, publisher, place and year, eventual page numbers
\end{itemize}

\noindent
\end{block_mdfboxadmon}
\end{frame}

\end{document}
